\chapter{Технологический раздел}

В данном разделе будут приведены  средства реализации и листинг кода.


\section{Средства реализации}

Для реализации данной работы был выбран язык \textit{C++}~\cite{cpp}.
Данный выбор обусловлен следующим:
\begin{itemize}
	\item язык поддерживает все структуры данных, которые выбраны в результате проектирования;
	\item язык позволяет реализовать все алгоритмы, выбранные в результате проектирования;
	\item язык позволяет работать с нативными потоками~\cite{thread}. 
\end{itemize}

Время выполнения реализаций было замерено с помощью функции \textit{clock}~\cite{clock}. 
Для хранения слов использовалась структура данных \textit{wstring}~\cite{wstring}, в качестве массивов использовалась структура данных \textit{vector}~\cite{vector}.
В качестве примитива синхронизации использовался \textit{mutex}~\cite{mutex}.

Для создания потоков и работы с ними был использован класс \textit{thread} из стандартной библиотеки выбранного языка~\cite{thread}.
В листинге \ref{lst:thead-example.cpp}, приведен пример работы с описанным классом, каждый объект класса представляет собой поток операционной системы, что позволяет нескольким функциям выполняться параллельно~\cite{thread}. 

\clearpage
\includelistingpretty
{thead-example.cpp} % Имя файла с расширением (файл должен быть расположен в директории inc/lst/)
{c++} % Язык программирования (необязательный аргумент)
{Пример работы с классом thread} % Подпись листинга



\section{Сведения о модулях программы}

Данная программа разбита на следующие модули:
\begin{itemize}
	\item $main.cpp$ --- файл, содержащий функцию $main$;
	\item $correcter.cpp$ --- файл, содержащий код реализации алгоритма исправления ошибок;
	\item $utils.cpp$ --- файл, в котором содержатся вспомогательные функции;
	\item $conveyor.cpp$ --- файл, в котором содержатся реализации элементов конвейера;
	\item $levenstein.cpp$ --- файл, в котором содержится реализация алгоритма поиска расстояния Левенштейна.
\end{itemize}

\section{Реализация алгоритмов}

В листинге \ref{lst:algomain.cpp} приведена реализация алгоритма исправления ошибок без дополнительных потоков. 
В листинге \ref{lst:runpipe.cpp} приведена реализация алгоритма запуска конвейера.
В листингах \ref{lst:device1.cpp} -- \ref{lst:device3.cpp} приведены реализации обслуживающих устройств.

\clearpage
\includelistingpretty
{algomain.cpp} % Имя файла с расширением (файл должен быть расположен в директории inc/lst/)
{python} % Язык программирования (необязательный аргумент)
{Функция исправления ошибок} % Подпись листинга

\clearpage

\includelistingpretty
{runpipe.cpp} % Имя файла с расширением (файл должен быть расположен в директории inc/lst/)
{python} % Язык программирования (необязательный аргумент)
{Функция запуска конвейера} % Подпись листинга

\clearpage

\includelistingpretty
{device1.cpp} % Имя файла с расширением (файл должен быть расположен в директории inc/lst/)
{python} % Язык программирования (необязательный аргумент)
{Функция обслуживающего устройства, которое проверяет содержится ли слово в корпусе} % Подпись листинга

\clearpage

\includelistingpretty
{device2.cpp} % Имя файла с расширением (файл должен быть расположен в директории inc/lst/)
{python} % Язык программирования (необязательный аргумент)
{Функция обслуживающего устройства, которое находит ближайшие слова} % Подпись листинга

\clearpage


\includelistingpretty
{device3.cpp} % Имя файла с расширением (файл должен быть расположен в директории inc/lst/)
{python} % Язык программирования (необязательный аргумент)
{Функция обслуживающего устройства, которое записывает слова в файл} % Подпись листинга




\section*{Вывод}
Были разработаны спроектированные алгоритмы исправления ошибок.
