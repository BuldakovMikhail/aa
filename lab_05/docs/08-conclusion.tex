\chapter*{ЗАКЛЮЧЕНИЕ}
\addcontentsline{toc}{chapter}{ЗАКЛЮЧЕНИЕ}

Цель данной лабораторной работы была достигнута, а именно описаны принципы конвейерных вычислений на основе нативных потоков для исправления орфографических ошибок в тексте

Для достижения поставленной цели были выполнены следующие задачи:
\begin{itemize}
	\item описан алгоритм исправления орфографических ошибок в тексте;
	\item спроектировано программное обеспечение, реализующее алгоритм и его конвейерную версию;
	\item выбраны инструменты для реализации и замера процессорного времени
	выполнения реализаций алгоритмов;
	\item проанализированы затраты реализаций алгоритмов по времени.
\end{itemize}


В результате исследования реализаций было получено, было получено, что время выполнения поточной реализации в 2 раза быстрее линейной реализации, при количестве заявок 70. Такой результат объясняется тем, что в поточной реализации потоки могут выполнять различные этапы работы параллельно, что позволяет сократить время обработки последовательности заявок.
