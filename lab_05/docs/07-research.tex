\chapter{Исследовательский раздел}

В данном разделе будут приведены: пример работы программы, постановка исследования и сравнительный анализ алгоритмов на основе полученных данных.

\section{Демонстрация работы программы}


На рисунке \ref{img:program} представлена демонстрация работы разработанного программного обеспечения, а именно показаны результаты генерации и конвейерной обработки 5 слов.  

\includeimage
{program} % Имя файла без расширения (файл должен быть расположен в директории inc/img/)
{f} % Обтекание (без обтекания)
{H} % Положение рисунка (см. figure из пакета float)
{0.7\textwidth} % Ширина рисунка
{Демонстрация работы программы} % Подпись рисунка

\clearpage

Технические характеристики устройства, на котором выполнялись замеры по времени, следующие:
\begin{itemize}
	\item процессор: AMD Ryzen 5 4600H 3 ГГц, 6 физических ядер, 12 логических процессоров~\cite{amd};
	\item оперативная память: 16 ГБайт;
	\item операционная система: Windows 10 Pro 64-разрядная система версии 22H2~\cite{windows}.
\end{itemize}

При замерах времени ноутбук был включен в сеть электропитания и был нагружен только системными приложениями.

\section{Время выполнения реализаций алгоритмов}

Результаты замеров времени выполнения реализаций приведены в таблице~\ref{tbl:time_measurements}.
Замеры времени проводились случайно сгенерированных словах. Замеры времени проводились на корпусах одного размера.

\begin{table}[H]
	\begin{center}
		\begin{threeparttable}
			\captionsetup{justification=raggedright,singlelinecheck=off}
			\caption{Время работы реализаций алгоритмов (в с)}
			\label{tbl:time_measurements}
			\begin{tabular}{|c|c|c|}
				\hline
				Количество слов & Поточная (с) & Линейная (с) \\
			\hline
			10 &$ 1.411\cdot 10^{-2} $&$ 3.488\cdot 10^{-2}$\\
			\hline
			20 &$ 2.715\cdot 10^{-2} $&$ 5.926\cdot 10^{-2}$\\
			\hline
			30 &$ 4.851\cdot 10^{-2} $&$ 9.296\cdot 10^{-2}$\\
			\hline
			40 &$ 6.160\cdot 10^{-2} $&$ 1.224\cdot 10^{-1}$\\
			\hline
			50 &$ 7.296\cdot 10^{-2} $&$ 1.292\cdot 10^{-1}$\\
			\hline
			60 &$ 9.610\cdot 10^{-2} $&$ 1.836\cdot 10^{-1}$\\
			\hline
			70 &$ 1.428\cdot 10^{-1} $&$ 2.461\cdot 10^{-1}$\\
			\hline
			\end{tabular}
		\end{threeparttable}
	\end{center}
\end{table}


На рисунке~\ref{img:measure} приведен график зависимости времени выполнения реализаций от количества заявок. 

\includeimage
{measure} % Имя файла без расширения (файл должен быть расположен в директории inc/img/)
{f} % Обтекание (без обтекания)
{H} % Положение рисунка (см. figure из пакета float)
{1\textwidth} % Ширина рисунка
{График зависимости времени выполнения реализации от количества заявок} % Подпись рисунка

\section*{Вывод}

В результате анализа таблицы \ref{tbl:time_measurements}, было получено, что время выполнения поточной реализации в 2 раза быстрее линейной реализации, при количестве заявок 70. Такой результат объясняется тем, что в поточной реализации потоки могут выполнять различные этапы работы параллельно, что позволяет сократить время обработки последовательности заявок.
