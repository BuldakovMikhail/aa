\chapter{Аналитический раздел}

В данном разделе будет представлена информация о многопоточности и исследуемом алгоритме исправления орфографических ошибок в тексте.

\section{Многопоточность} 

Многопоточность --- это способность центрального процессора одновременно выполнять несколько потоков, используя ресурсы одного процессора. Каждый поток представляет собой последовательность инструкций, которые могут выполняться параллельно с другими потоками, созданными одним и тем же процессом~\cite{multithreading}.

Процессом называют программу в стадии выполнения \cite{process}. Один процесс может иметь один или несколько потоков. Поток --- это часть процесса, которая выполняет задачи, необходимые для выполнения приложения. Процесс завершается, когда все его потоки полностью завершены. 

Одной из сложностей, связанных с использованием потоков, является проблема доступа к данным. 
Основным ограничением является невозможность одновременной записи в одну и ту же ячейку памяти из разных потоков.
Это означает, что нужен механизм синхронизации доступа к данным, так называемый <<мьютекс>> (от англ. mutex --- mutual exclusion, взаимное исключение). 
Мьютекс может быть захвачен одним потоком для работы в режиме монопольного использования или освобожден. Если два потока попытаются захватить мьютекс одновременно, то успех будет у одного потока, а другой будет блокирован, пока мьютекс не освободится.

\section{Исправления орфографических ошибок в тексте}

Для распознавания слов, написанных с ошибками, используется расстояние
Левенштейна --- минимальное количество ошибок, исправление которых приводит
одно слово к другому~\cite{miem}. 
Т.~о. для введенного слова осуществляется проверка по корпусу, если данное слово не найдено в корпусе, то ищется ближайшее слово к данному по расстоянию Левенштейна. 

Кроме того, следует вводить ограничение на количество ошибок, которые позволяется допустить. Как говорит поговорка: «Если в слове хлеб допустить всего четыре ошибки, то получится слово пиво»~\cite{miem}. 
Если фиксируется число ошибок, то для коротких слов оно может оказаться избыточным.
Верхнюю границу числа ошибок обычно ограничивают как процентным соотношением, так и фиксированным числом. 
Например, не более $30\%$ букв входного слова, но не более 3.
При этом все равно стараются найти слова с минимальным количеством ошибок \cite{miem}.

\section{Использование потоков для исправления орфографических ошибок}

Поскольку задача сводится к поиску слова в корпусе, можно распараллелить поиск по этому корпусу. В таком случае каждый поток будет вычислять расстояние Левенштейна между заданным словом и некоторым словом из корпуса и в случае, если расстояние будет удовлетворять требованиям, то данное слово будет записано в массив. При записи подходящих слов в массив, возможна ситуация, когда значение длины массива считывается в одном потоке и в тот же момент изменяется в другом потоке, т.~е. возникает конфликт. Для решения проблем синхронизации необходимо использовать мьютекс, чтобы обеспечить монопольный доступ к длине массива. 

\section*{Вывод} 
В данном разделе была представлена информация о многопоточности
и исследуемом алгоритме.