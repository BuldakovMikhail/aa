\chapter{Исследовательский раздел}

В данном разделе будут приведены примеры работы программ, постановка эксперимента и сравнительный анализ алгоритмов на основе полученных данных.


\section{Технические характеристики}

Технические характеристики устройства, на котором выполнялись замеры по времени:

\begin{itemize}
	\item Процессор: AMD Ryzen 5 4600H 3 ГГц \cite{amd}.
	\item Оперативная память: 16 ГБайт.
	\item Операционная система: Windows 10 Pro 64-разрядная система версии 22H2 \cite{windows}.
\end{itemize}

При замерах времени ноутбук был включен в сеть электропитания и был нагружен только системными приложениями.


\section{Демонстрация работы программы}

%Тут абзац со ссылкой на рисунок и описанием того, что на нём представлено/происходит
На рисунке \ref{img:program} представлена демонстрация работы разработанного программного обеспечения, а именно показаны результаты вычислений расстояний Левенштейна и Дамерау---Левенштейна для строк <<скат>>, <<кот>> и <<красивый>>, <<карсивый>> соответственно.  
\clearpage

\includeimage
{program} % Имя файла без расширения (файл должен быть расположен в директории inc/img/)
{f} % Обтекание (без обтекания)
{h} % Положение рисунка (см. figure из пакета float)
{1\textwidth} % Ширина рисунка
{Демонстрация работы программы при поиске расстояний Левенштейна и Дамерау---Левенштейна} % Подпись рисунка

\clearpage

\section{Время выполнения алгоритмов}

Результаты замеров времени работы алгоритмов нахождения расстояний
Левенштейна и Дамерау–Левенштейна приведены в таблице \ref{tbl:time_measurements}. Замеры времени проводились на строках одинаковой длины и усреднялись для каждого набора одинаковых экспериментов.

\begin{table}[h]
	\begin{center}
		\begin{threeparttable}
			\captionsetup{justification=raggedright,singlelinecheck=off}
			\caption{Время работы алгоритмов (в секундах)}
			\label{tbl:time_measurements}
			\begin{tabular}{|c|c|c|c|c|}
				\hline
				Длина строк &  Л (и)  & Д-Л (и) & Д-Л (р) & Д-Л (рк) \\
				\hline
				1 & 0.0 & 7812.5 & 0.0 & 0.0 \\
				\hline
				2 & 0.0 & 7812.5 & 0.0 & 0.0 \\
				\hline
				3 & 7812.5 & 7812.5 & 78125.0 & 0.0 \\
				\hline
				4 & 7812.5 & 15625.0 & 234375.0 & 0.0 \\
				\hline
				5 & 15625.0 & 15625.0 & 1093750.0 & 78125.0 \\
				\hline
				6 & 23437.5 & 23437.5 & 5234375.0 & 78125.0 \\
				\hline
				7 & 39062.5 & 39062.5 & 27734375.0 & 78125.0 \\
				\hline
				8 & 39062.5 & 46875.0 & 151015625.0 & 78125.0 \\
				\hline
				9 & 46875.0 & 54687.5 & 833437500.0 & 156250.0 \\
				\hline
				10 & 62500.0 & 70312.5 & 4610859375.0 & 156250.0 \\
				\hline
				
				
			\end{tabular}
		\end{threeparttable}
	\end{center}
\end{table}

\includeimage
{Figure2} % Имя файла без расширения (файл должен быть расположен в директории inc/img/)
{f} % Обтекание (без обтекания)
{h} % Положение рисунка (см. figure из пакета float)
{1\textwidth} % Ширина рисунка
{Сравнение алгоритмов по времени} % Подпись рисунка

\clearpage

Наиболее эффективными являются алгоритмы, использующие матрицы, после них по скорости работы идет алгоритм использующий кеш, это обусловлено тем, что по сравнению с обычной рекурсией, мы не вычисляем повторно одни и те же значения, но все равно тратим время на рекурсивные вызовы. 

\section{Характеристики по памяти}

\label{memory}

Введем следующие обозначения:
\begin{itemize}
	\item $n$ -- длина строки $S_1$;
	\item $m$ -- длина строки $S_2$;
	\item $size()$ -- функция, вычисляющая размер в байтах;
	\item $int$ -- целочисленный тип данных;
	\item $string$ -- строковый тип данных.
\end{itemize}

Т.~к. алгоритмы, вычисляющие расстояния Левенштейна и Дамерау---Левенштейна, не отличаются по использованию памяти, то достаточно рассмотреть итеративную, рекурсивную и рекурсивную с кешированием реализации алгоритмов вычисления расстояния Дамерау---Левенштейна.


Использование памяти при итеративной реализации теоритически рассчитывется по формул \eqref{eq:iter_mem}.
\begin{equation}
	\label{eq:iter_mem}
	(n + 1) * (m + 1) * size(int) + 2 * size(string) + 2 * size(int),
\end{equation}
где 
\begin{itemize}
	\item $ (n + 1) * (m + 1) * size(int) $ -- хранение матрицы;
	\item $ 2 * size(string) $ -- хранение двух строк;
	\item $ 2 * size(int) $ -- адрес возврата и возвращаемое значение.
\end{itemize}


Максимальная глубина стека вызовов при рекурсивной реализации
нахождения расстояния Дамерау---Левенштейна равна сумме входящих строк,
соответственно, максимальный расход памяти рассчитывается по \eqref{eq:rec_mem}.

\begin{equation}
	\label{eq:rec_mem}
	(n + m) * (2 * size(string) + 3 * size(int)),
\end{equation}
где 
\begin{itemize}
	\item $ (n + m) $ -- максимальная глубина стека вызовов;
	\item $ 2 * size(string) $ -- хранение двух строк;
	\item $ 2 * size(int) $ -- адрес возврата и возвращаемое значение;
	\item $ size(int) $ -- временная переменная.
\end{itemize}

Для алгоритма, использующего кеширование требуется дополнительно память под кеш и 4 временных переменных \eqref{eq:req_cash_mem}.

\begin{equation}
	\label{eq:req_cash_mem}
	(n + m) * (2 * size(string) + 6 * size(int)) + (n + 1) * (m + 1) * size(int),
\end{equation}
где 
\begin{itemize}
	\item $ (n + m) $ -- максимальная глубина стека вызовов;
	\item $ 2 * size(string) $ -- хранение двух строк;
	\item $ 2 * size(int) $ -- адрес возврата и возвращаемое значение;
	\item $ 4 * size(int) $ -- временные переменные;
	\item $ (n + 1) * (m + 1) * size(int) $ -- хранение кеша.
\end{itemize}

По расходу памяти итеративные алгоритмы проигрывают рекурсивным: максимальный размер используемой памяти в итеративном растет
как произведение длин строк, в то время как у рекурсивного алгоритма --
как сумма длин строк.


\section*{Вывод}

В данном разделе было произведено сравнение количества затраченного времени и памяти алгоритмов поиска расстояний Левенштейна и
Дамерау---Левенштейна. Наименее затратным по времени оказался итеративный алгоритм нахождения расстояния Левенштейна.

По таблице \ref{tbl:time_measurements} видно, что рекурсивный алгоритм в 65577 раз проигрывает итеративному при длине строк 10. Поэтому рекурсивные алгоритмы следует использовать лишь при малых длинах строк.

При этом как было замечено в пункте \ref{memory}, рекурсивные алгоритмы занимают меньше памяти, чем итеративные алгоритмы.

Рекурсивная реализация алгоритма поиска расстояния Дамерау---Левенштейн будет более затратным по времени по сравнению с итеративной реализацией алгоритма поиска расстояния Дамерау---Левенштейна, но менее затратным по памяти по отношению к итеративному алгоритму Дамерау---Левенштейна. При этом рекурсивные алгоритм с кешированием проигрывает по памяти и по времени итеративному.