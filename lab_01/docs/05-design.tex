\chapter{Конструкторский раздел}


В этом разделе будут приведены требования к вводу и программе, а также схемы алгоритмов нахождения расстояний Левенштейна и Дамерау---Левенштейна.

\section{Требования к программному обеспечению}

К программе предъявлен ряд требований:
\begin{itemize}
	\item на вход подаются две строки, которые могут быть пустыми;
	\item на выходе --- результат работы всех алгоритмов поиска расстояний, целое число;
	\item наличие интерфейса для выбора действий;
	\item возможность обработки строк, включающих буквы как на латинице, так и на кириллице;
	\item возможность замерить процессорное время работы реализаций алгоритмов поиска расстояний Левенштейна и Дамерау---Левенштейна.
\end{itemize}


\section{Разработка алгоритмов}

На вход алгоритмов подаются строки $S1$ и $S2$.

На рисунке \ref{img:lev2} представлена схема алгоритма поиска расстояния Левенштейна. На рисунках \ref{img:DLrec} -- \ref{img:DLiter} представлены схемы алгоритмов поиска расстояния Дамерау---Левенштейна.

\includeimage
{lev2} % Имя файла без расширения (файл должен быть расположен в директории inc/img/)
{f} % Обтекание (без обтекания)
{h} % Положение рисунка (см. figure из пакета float)
{0.9\textwidth} % Ширина рисунка
{Нерекурсивный алгоритм нахождения расстояния Левенштейна} % Подпись рисунка

\includeimage
{DLrec} % Имя файла без расширения (файл должен быть расположен в директории inc/img/)
{f} % Обтекание (без обтекания)
{h} % Положение рисунка (см. figure из пакета float)
{1\textwidth} % Ширина рисунка
{Рекурсивный алгоритм нахождения расстояния Дамерау---Левенштейна} % Подпись рисунка

\includeimage
{DLrecCash} % Имя файла без расширения (файл должен быть расположен в директории inc/img/)
{f} % Обтекание (без обтекания)
{h} % Положение рисунка (см. figure из пакета float)
{1\textwidth} % Ширина рисунка
{Рекурсивный алгоритм нахождения расстояния Дамерау---Левенштейна с кешем} % Подпись рисунка

\includeimage
{DLrecCash21} % Имя файла без расширения (файл должен быть расположен в директории inc/img/)
{f} % Обтекание (без обтекания)
{h} % Положение рисунка (см. figure из пакета float)
{0.8\textwidth} % Ширина рисунка
{Функция, заполняющая матрицу расстояний Дамерау---Левенштейна рекурсивно} % Подпись рисунка

\includeimage
{DLRecCash3} % Имя файла без расширения (файл должен быть расположен в директории inc/img/)
{f} % Обтекание (без обтекания)
{h} % Положение рисунка (см. figure из пакета float)
{1\textwidth} % Ширина рисунка
{Функция, извлекающая и записывающая значения кеш-матрицы} % Подпись рисунка


\includeimage
{DLiter} % Имя файла без расширения (файл должен быть расположен в директории inc/img/)
{f} % Обтекание (без обтекания)
{h} % Положение рисунка (см. figure из пакета float)
{0.8\textwidth} % Ширина рисунка
{Нерекурсивный алгоритм нахождения расстояния Дамерау---Левенштейна} % Подпись рисунка

\clearpage

\section{Описание используемых типов и структур данных}

Для реализации алгоритмов, будут использованы следующие типы данных:
\begin{itemize}
	\item \textit{str} --- для двух строк, поданных на вход;
	\item \textit{int} --- для возвращаемого значения искомого расстояния.
\end{itemize}

При реализации алгоритмов будет использована структура данных --- матрица, которая является двумерным списком значений типа \textit{int}.


\section*{Вывод}

В данном разделе на основе теоретических данных были построены схемы требуемых алгоритмов, выбраны используемые типы данных.
