\chapter{Конструкторский раздел}


В этом разделе будут приведены требования к вводу и программе, а также схемы алгоритмов нахождения расстояний Левенштейна и Дамерау---Левенштейна.

\section{Требования к вводу}
\begin{enumerate}
	\item На вход подаются две строки, которые могут быть пустыми.
	\item Буквы верхнего и нижнего регистров считаются различными.
\end{enumerate}

\section{Требования к программе}
\begin{enumerate}
	\item Обрабатывать корректно любые входные строки.
	\item В результате программа должна вывести число -- расстояние Левенштейна (Дамерау---Левенштейна).
	\item Возможность обработки строк, включающих буквы как на латинице, так и на кириллице.
	\item Возможность замерить процессорное время работы реализаций алгоритмов поиска расстояний Левенштейна и Дамерау---Левенштейна.
\end{enumerate}


\section{Разработка алгоритмов}

На вход алгоритмов подаются строки $S1$ и $S2$.

На рисунке \ref{img:lev2} представлена схема алгоритма поиска расстояния Левенштейна. На рисунках \ref{img:DLrec} -- \ref{img:DLiter} представлены схемы алгоритмов поиска расстояния Дамерау---Левенштейна.

\includeimage
{lev2} % Имя файла без расширения (файл должен быть расположен в директории inc/img/)
{f} % Обтекание (без обтекания)
{h} % Положение рисунка (см. figure из пакета float)
{0.9\textwidth} % Ширина рисунка
{Нерекурсивный алгоритм нахождения расстояния Левенштейна} % Подпись рисунка

\includeimage
{DLrec} % Имя файла без расширения (файл должен быть расположен в директории inc/img/)
{f} % Обтекание (без обтекания)
{h} % Положение рисунка (см. figure из пакета float)
{1\textwidth} % Ширина рисунка
{Рекурсивный алгоритм нахождения расстояния Дамерау---Левенштейна} % Подпись рисунка

\includeimage
{DLrecCash} % Имя файла без расширения (файл должен быть расположен в директории inc/img/)
{f} % Обтекание (без обтекания)
{h} % Положение рисунка (см. figure из пакета float)
{1\textwidth} % Ширина рисунка
{Рекурсивный алгоритм нахождения расстояния Дамерау---Левенштейна с кешем} % Подпись рисунка

\includeimage
{DLrecCash21} % Имя файла без расширения (файл должен быть расположен в директории inc/img/)
{f} % Обтекание (без обтекания)
{h} % Положение рисунка (см. figure из пакета float)
{1\textwidth} % Ширина рисунка
{Функция заполняющая матрицу расстояний Дамерау---Левенштейна рекурсивно} % Подпись рисунка


\includeimage
{DLiter} % Имя файла без расширения (файл должен быть расположен в директории inc/img/)
{f} % Обтекание (без обтекания)
{h} % Положение рисунка (см. figure из пакета float)
{0.8\textwidth} % Ширина рисунка
{Нерекурсивный алгоритм нахождения расстояния Дамерау---Левенштейна} % Подпись рисунка


\section{Описание используемых типов и структур данных}

Для реализации алгоритмов, будут использованы следующие типы данных:
\begin{itemize}
	\item \textit{str} -- для двух строк, поданных на вход;
	\item \textit{int} -- для возвращаемого значения искомого расстояния.
\end{itemize}

При реализации алгоритмов будет использована структура данных -- матрица, которая является двумерным списком значений типа \textit{int}.


\section*{Вывод}

В данном разделе на основе теоретических данных были построены схемы требуемых алгоритмов, выбраны используемые типы данных.
