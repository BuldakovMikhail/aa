\chapter*{ВВЕДЕНИЕ}
\addcontentsline{toc}{chapter}{ВВЕДЕНИЕ}

Расстояние Левенштейна -- минимальное количество редакционных операций, которое необходимо выполнить для преобразования одной строки в другую \cite{levenshtein}. Редакционными операциями являются: 
\begin{itemize}
	\item I -- вставка одного символа (insert);
	\item M -- удаление (match);
	\item R -- замена (replace).
\end{itemize}
Также обозначим совпадение как M (match).

Расстояние Дамерау---Левенштейна является модификацией расстояния Левенштейна, отличается от него добавлением операции транспозиции (перестановки).  

Редакционные расстояния применяются для решения следующих задач:
\begin{itemize}
	\item исправление ошибок в словах;
	\item обучение языковых моделей (расстояние Левенштейна вводится как метрика);
	\item сравнение геномов, хромосом и белков в биоинформатике.
\end{itemize}

Целью данной лабораторной работы является исследование алгоритмов вычисляющих расстояние Левенштейна и Дамерау---Левенштейна.

Для достижения поставленной цели необходимо выполнить следующие задачи:
\begin{enumerate}
	\item Изучить алгоритмы, вычисляющие расстояния Левенштейна и Дамерау---Левенштейна.
	\item Разработать программное обеспечение, реализующее следующие алгоритмы:
	\begin{itemize}
		\item нерекурсивный алгоритм поиска расстояния Дамерау---Левенштейна;
		\item рекурсивный алгоритм поиска расстояния Дамерау---Левенштейна без кеширования;
		\item рекурсивный алгоритм поиска расстояния Дамерау---Левенштейна с кешированием;
		\item нерекурсивный алгоритм поиска расстояния Левенштейна.
	\end{itemize}
	\item Выбрать инструменты для реализации и замера процессорного времени выполнения алгоритмов, описанных выше.
	\item Проанализировать затраты реализаций алгоритмов по времени и по памяти.
\end{enumerate}
