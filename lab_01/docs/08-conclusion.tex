\chapter*{ЗАКЛЮЧЕНИЕ}
\addcontentsline{toc}{chapter}{ЗАКЛЮЧЕНИЕ}

Цель данной лабораторной работы была достигнута, а именно были исследованы алгоритмы, вычисляющие расстояния Левенштейна и Дамерау---Левенштейна.

Для достижения поставленной цели были выполнены следующие задачи.
\begin{enumerate}
	\item Описаны алгоритмы поиска расстояний Левенштейна и \newline Дамерау---Левенштейна;
	\item Выбраны инструменты для реализации алгоритмов и замера процессорного времени их выполнения.
	\item Разработано программное обеспечение, реализующее следующие алгоритмы:
	\begin{itemize}
		\item нерекурсивный метод поиска расстояния Левенштейна;
		\item нерекурсивный метод поиска расстояния Дамерау---Левенштейна;
		\item рекурсивный метод поиска расстояния Дамерау---Левенштейна;
		\item рекурсивный с кешированием метод поиска расстояния Дамерау---Левенштейна.
	\end{itemize}
	\item Проведен анализ затрат реализаций алгоритмов по времени и по памяти. 
\end{enumerate}

В результате исследования реализаций алгоритмов было выявлено, что рекурсивная реализация в 72477 раз проигрывает итеративной по времени выполнения при длине строк 10. 
Это обусловлено тем, что рекурсивная реализация не хранит вычисленные значения расстояний для подстрок и поэтому вычисляет их множество раз. 
При этом рекурсивная реализация требует меньше памяти, чем итеративные, поэтому лучше подходит для коротких строк. 

Рекурсивная реализация алгоритма поиска расстояния Дамерау---Левенштейна с кешированием проигрывает в 3 раза итеративной реализации, но выигрывает в 27142 раза рекурсивную реализацию по времени выполнения при длине строк 10. 
Такой результат обуславливается тем, что в отличии от рекурсивной реализации ранее вычисленные значения сохраняются в памяти, но все равно затрачивается время на рекурсивные вызовы. 
В то же время реализация с кешированием проигрывает всем реализациям по памяти.

Итеративные реализации алгоритмов поиска расстояний Дамерау---Левенштейна и Левенштейна практически не различаются по времени выполнения.

