\chapter{Технологический раздел}

В данном разделе будут приведены требования к программному обеспечению, средства реализации, листинг кода и функциональные тесты.

\section{Средства реализации}

Для реализации данной работы был выбран язык \textit{Python}~\cite{python}. Такой выбор обусловлен опытом работы с этим языком программирования. Также данный язык позволяет замерять процессорное время с помощью модуля \textit{time}.

Время работы было замерено с помощью функции \textit{process\_time\_ns()} из модуля \textit{time}~\cite{python-time}.

\section{Сведения о модулях программы}

Данная программа разбита на следующие модули:
\begin{itemize}
	\item $main.py$ -- файл, содержащий функцию $main$;
	\item $algorithms.py$ -- файл, содержащий код всех алгоритмов нахождения расстояний Левенштейна и Дамерау---Левенштейна;
	\item $compare\_time.py$ -- файл, в котором содержатся функции для замера и вывода времени работы алгоритмов.
\end{itemize}

\section{Реализация алгоритмов}

В листингах \ref{lst:levmtr.py} -- \ref{lst:dameraylevreccash.py} приведены реализации алгоритмов поиска расстояний Левенштейна (только нерекурсивный алгоритм) и Дамерау---Левенштейна (нерекурсивный, рекурсивный, рекурсивный с кешированием).

\clearpage

\includelistingpretty
{levmtr.py} % Имя файла с расширением (файл должен быть расположен в директории inc/lst/)
{python} % Язык программирования (необязательный аргумент)
{Функция нахождения расстояния Левенштейна с использованием матрицы} % Подпись листинга

\clearpage

\includelistingpretty
{dameraylevmtr.py} % Имя файла с расширением (файл должен быть расположен в директории inc/lst/)
{python} % Язык программирования (необязательный аргумент)
{Функция нахождения расстояния Дамерау---Левенштейна с использованием матрицы} % Подпись листинга

\clearpage

\includelistingpretty
{dameraylevrec.py} % Имя файла с расширением (файл должен быть расположен в директории inc/lst/)
{python} % Язык программирования (необязательный аргумент)
{Функция нахождения расстояния Дамерау---Левенштейна рекурсивно} % Подпись листинга

\clearpage

\includelistingpretty
{dameraylevreccash.py} % Имя файла с расширением (файл должен быть расположен в директории inc/lst/)
{python} % Язык программирования (необязательный аргумент)
{Функция нахождения расстояния Дамерау---Левенштейна рекурсивно c кешированием} % Подпись листинга

\clearpage


\section{Функциональные тесты}

В таблице \ref{tbl:func_tests} приведены функциональные тесты для алгоритмов вычисления расстояний Левенштейна и Дамерау---Левенштейна. Все тесты пройдены успешно.

\begin{table}[ht]
	\small
	\begin{center}
		\begin{threeparttable}
			\caption{Функциональные тесты}
			\label{tbl:func_tests}
			\begin{tabular}{|c|c|c|c|c|c|}
				\hline
				\multicolumn{2}{|c|}{\bfseries Входные данные}
				& \multicolumn{4}{c|}{\bfseries Расстояние и алгоритм} \\ 
				\hline 
				&
				& \multicolumn{1}{c|}{\bfseries Левенштейна} 
				& \multicolumn{3}{c|}{\bfseries Дамерау---Левенштейна} \\ \cline{3-6}
				
				\bfseries Строка 1 & \bfseries Строка 2 & \bfseries Итеративный & \bfseries Итеративный
				
				& \multicolumn{2}{c|}{\bfseries Рекурсивный} \\ \cline{5-6}
				& & & & \bfseries Без кеша & \bfseries С кешом \\
				\hline
				$\lambda$ & $\lambda$ & 0 & 0 & 0 & 0 \\
				\hline
				a & b & 1 & 1 & 1 & 1 \\
				\hline
				a & a & 0 & 0 & 0 & 0 \\
				\hline
				кот & скат & 2 & 2 & 2 & 2 \\
				\hline
				ab & ba & 2 & 1 & 1 & 1 \\
				\hline
				bba & abba & 1 & 1 & 1 & 1 \\
				\hline
				aboba & boba & 1 & 1 & 1 & 1 \\
				\hline
				abcdef & gh & 6 & 6 & 6 & 6 \\
				\hline
				
			\end{tabular}	
		\end{threeparttable}
	\end{center}
\end{table}

\section*{Вывод}

Были реализованы алгоритмы поиска расстояния Левенштейна итеративно, поиска расстояния Дамерау–Левенштейна итеративно, рекурсивно и рекурсивно с кешированием. Проведено тестирование реализаций алгоритмов.
