\chapter*{ЗАКЛЮЧЕНИЕ}
\addcontentsline{toc}{chapter}{ЗАКЛЮЧЕНИЕ}

Цель данной лабораторной работы была достигнута, а именно описаны принципы параллельных вычислений на основе нативных потоков для исправления орфографических ошибок в тексте

Для достижения поставленной цели были выполнены следующие задачи:
\begin{itemize}
	\item описан алгоритм исправления орфографических ошибок в тексте;
	\item спроектировано программное обеспечение, реализующее алгоритм и его параллельную версию;
	\item выбраны инструменты для реализации и замера процессорного времени
	выполнения реализаций алгоритмов;
	\item проанализированы затраты реализаций алгоритмов по времени.
\end{itemize}


В результате исследования реализаций было получено, что время выполнения многопоточной реализации алгоритма меньше времени выполнения однопоточной реализации при количестве потоков не больше 6, т.~е. когда количество потоков не больше количества физических ядер. При двух дополнительных потока многопоточная реализации быстрее однопоточной в 2.28 раза.

При увеличении числа слов в корпусе время получения результата с использованием нескольких потоков также увеличивается. 
При увеличении размеров корпуса с
1000 до 10000, время получения результата без использования дополнительных потоков
увеличилось в 10 раз, а время получения результата с использованием одного потока увеличилось в 27 раз. 
При этом при 10000 слов в корпусе реализация с одним дополнительным потоком быстрее реализации без дополнительных потоков в 1.8 раза. 
Таким образом, рекомендуется использование
двух вспомогательных потоков, т.~к. это позволит осуществлять поиск по корпусу с минимальными тратами на переключения контекста.
