\chapter{Конструкторский раздел}

В этом разделе будет представлено описание используемых типов данных, а также схемы алгоритмов исправления орфографических ошибок.

\section{Требования к программному обеспечению}

Программа должна поддерживать два режима работы: режим массового замера времени и режим исправления введенного слова.

Режим массового замера времени должен обладать следующей функциональностью:
\begin{itemize}
	\item генерировать корпус слов;
	\item осуществлять массовый замер, используя сгенерированные данные;
	\item результаты массового замера должны быть представлены в виде таблицы и графика.
\end{itemize}

К режиму исправления введенного слова выдвигается следующий ряд требований:
\begin{itemize}
	\item возможность вводить слова, которые отсутствуют в корпусе;
	\item наличие интерфейса для выбора действий;
	\item на выходе программы, набор из самых близких к введенному слов.
\end{itemize}

\section{Описание используемых типов данных}

При реализации алгоритмов будут использованы следующие структуры
и типы данных:
\begin{itemize}
	\item слово --- массив букв;
	\item корпус --- массив слов, отсортированный в лексикографическом порядке;
	\item мьютекс --- примитив синхронизации.
\end{itemize}


\section{Разработка алгоритмов}

На рисунке \ref{img:algomain} представлена схема поиска ближайших слов в корпусе без использования потоков. 

\clearpage

\includeimage
{algomain} % Имя файла без расширения (файл должен быть расположен в директории inc/img/)
{f} % Обтекание (без обтекания)
{h} % Положение рисунка (см. figure из пакета float)
{1\textwidth} % Ширина рисунка
{Схема алгоритма гномьей сортировки} % Подпись рисунка

\clearpage
