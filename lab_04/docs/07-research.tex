\chapter{Исследовательский раздел}

В данном разделе будут приведены: пример работы программы, постановка исследования и сравнительный анализ алгоритмов на основе полученных данных.

\section{Демонстрация работы программы}


На рисунках \ref{img:program} и \ref{img:program2} представлена демонстрация работы разработанного программного обеспечения, а именно показаны результаты исправления ошибок в слове \textit{policu}.  

\includeimage
{program} % Имя файла без расширения (файл должен быть расположен в директории inc/img/)
{f} % Обтекание (без обтекания)
{H} % Положение рисунка (см. figure из пакета float)
{0.7\textwidth} % Ширина рисунка
{Демонстрация работы программы при исправлении слова без потоков} % Подпись рисунка

\includeimage
{program2} % Имя файла без расширения (файл должен быть расположен в директории inc/img/)
{f} % Обтекание (без обтекания)
{H} % Положение рисунка (см. figure из пакета float)
{0.7\textwidth} % Ширина рисунка
{Демонстрация работы программы при исправлении слова c дополнительными потоками} % Подпись рисунка


Технические характеристики устройства, на котором выполнялись замеры по времени, следующие:
\begin{itemize}
	\item процессор: AMD Ryzen 5 4600H 3 ГГц, 6 физических ядер, 12 логических процессоров~\cite{amd};
	\item оперативная память: 16 ГБайт;
	\item операционная система: Windows 10 Pro 64-разрядная система версии 22H2~\cite{windows}.
\end{itemize}

При замерах времени ноутбук был включен в сеть электропитания и был нагружен только системными приложениями.

\section{Время выполнения реализаций алгоритмов}

Результаты замеров времени выполнения реализации алгоритма исправления ошибок в зависимости от числа потоков приведены в таблице \ref{tbl:time_measurements}.
Замеры времени проводились на корпусе длины 10000, состоящего из слов длины 10. Замеры времени проводились на корпусах одного размера и усреднялись для каждого набора одинаковых экспериментов.

\begin{table}[H]
	\begin{center}
		\begin{threeparttable}
			\captionsetup{justification=raggedright,singlelinecheck=off}
			\caption{Время работы реализации алгоритмов решения задачи коммивояжера (в с)}
			\label{tbl:time_measurements}
			\begin{tabular}{|c|c|c|}
				\hline
				Количество городов &  Полный перебор  & Муравьиный \\
				\hline
				4 &$ 1.563\cdot10^{-5} $&$ 1.422\cdot10^{-3}$\\
				\hline
				5 &$ 4.688\cdot10^{-5} $&$ 3.125\cdot10^{-3}$\\
				\hline
				6 &$ 2.344\cdot10^{-4} $&$ 5.750\cdot10^{-3}$\\
				\hline
				7 &$ 1.594\cdot10^{-3} $&$ 1.005\cdot10^{-2}$\\
				\hline
				8 &$ 1.231\cdot10^{-2} $&$ 1.664\cdot10^{-2}$\\
				\hline
				9 &$ 1.105\cdot10^{-1} $&$ 2.506\cdot10^{-2}$\\
				\hline
			\end{tabular}
		\end{threeparttable}
	\end{center}
\end{table}
	