\chapter{Исследовательский раздел}

В данном разделе будут приведены: пример работы программы, постановка исследования и сравнительный анализ алгоритмов на основе полученных данных.

\section{Демонстрация работы программы}


На рисунках \ref{img:program} и \ref{img:program2} представлена демонстрация работы разработанного программного обеспечения, а именно показаны результаты исправления ошибок в слове \textit{policu}.  

\includeimage
{program} % Имя файла без расширения (файл должен быть расположен в директории inc/img/)
{f} % Обтекание (без обтекания)
{H} % Положение рисунка (см. figure из пакета float)
{0.7\textwidth} % Ширина рисунка
{Демонстрация работы программы при исправлении слова без потоков} % Подпись рисунка

\includeimage
{program2} % Имя файла без расширения (файл должен быть расположен в директории inc/img/)
{f} % Обтекание (без обтекания)
{H} % Положение рисунка (см. figure из пакета float)
{0.7\textwidth} % Ширина рисунка
{Демонстрация работы программы при исправлении слова c дополнительными потоками} % Подпись рисунка


Технические характеристики устройства, на котором выполнялись замеры по времени, следующие:
\begin{itemize}
	\item процессор: AMD Ryzen 5 4600H 3 ГГц, 6 физических ядер, 12 логических процессоров~\cite{amd};
	\item оперативная память: 16 ГБайт;
	\item операционная система: Windows 10 Pro 64-разрядная система версии 22H2~\cite{windows}.
\end{itemize}

При замерах времени ноутбук был включен в сеть электропитания и был нагружен только системными приложениями.

\section{Время выполнения реализаций алгоритмов}

Результаты замеров времени выполнения реализации алгоритма исправления ошибок в зависимости от числа потоков приведены в таблице~\ref{tbl:time_measurements}.
Замеры времени проводились на корпусе длины 10000, состоящего из слов длины 10. Замеры времени проводились на корпусах одного размера и усреднялись для каждого набора одинаковых экспериментов.

\begin{table}[H]
	\begin{center}
		\begin{threeparttable}
			\captionsetup{justification=raggedright,singlelinecheck=off}
			\caption{Время работы реализации алгоритма исправления ошибок (в мс)}
			\label{tbl:time_measurements}
			\begin{tabular}{|c|c|}
				\hline
				Количество потоков &  Время исправления ошибки (мс) \\
				\hline
		1 &$ 3.996$\\
		\hline
		2 &$ 3.208$\\
		\hline
		3 &$ 4.192$\\
		\hline
		4 &$ 5.388$\\
		\hline
		5 &$ 6.712$\\
		\hline
		6 &$ 7.061$\\
		\hline
		7 &$ 8.252$\\
		\hline
		9 &$ 8.382$\\
		\hline
		11 &$ 8.500$\\
		\hline
		13 &$ 8.368$\\
		\hline
		15 &$ 8.615$\\
		\hline
		17 &$ 9.571$\\
		\hline
		19 &$ 9.775$\\
		\hline
		21 &$ 10.24$\\
		\hline
		23 &$ 11.45$\\
		\hline
		25 &$ 12.20$\\
		\hline
		27 &$ 12.51$\\
		\hline
		29 &$ 13.28$\\
		\hline
		31 &$ 14.10$\\
		\hline
		33 &$ 14.81$\\
		\hline
		35 &$ 15.16$\\
		\hline
		37 &$ 15.47$\\
		\hline
		39 &$ 16.83$\\
		\hline
		41 &$ 17.32$\\
		\hline
		43 &$ 17.99$\\
		\hline
		45 &$ 18.84$\\
		\hline
		47 &$ 19.69$\\
		\hline
		49 &$ 20.84$\\
		\hline
		
			\end{tabular}
		\end{threeparttable}
	\end{center}
\end{table}

\clearpage

Результаты замеров времени выполнения однопоточной и многопоточной реализаций алгоритма исправления ошибок в зависимости от размеров корпуса приведены в таблице~\ref{tbl:time_measurements2}. Замеры времени проводились на корпусах одного размера и усреднялись для каждого набора одинаковых экспериментов.


\begin{table}[H]
	\begin{center}
		\begin{threeparttable}
			\captionsetup{justification=raggedright,singlelinecheck=off}
			\caption{Время работы реализаций алгоритма исправления ошибок (в мс)}
			\label{tbl:time_measurements2}
			\begin{tabular}{|c|c|c|}
				\hline
				Размер корпуса &  C доп. потоком (мс) & Без доп. потоков (мс)\\
				\hline
			1000 &$ 1.536\cdot 10^{-1} $&$ 7.337\cdot 10^{-1}$\\
			\hline
			2000 &$ 3.722\cdot 10^{-1} $&$ 1.472$\\
			\hline
			3000 &$ 6.929\cdot 10^{-1} $&$ 2.162$\\
			\hline
			4000 &$ 1.015 $&$ 2.934$\\
			\hline
			5000 &$ 1.173 $&$ 3.675$\\
			\hline
			6000 &$ 1.831 $&$ 4.418$\\
			\hline
			7000 &$ 2.147 $&$ 5.187$\\
			\hline
			8000 &$ 2.715 $&$ 5.985$\\
			\hline
			9000 &$ 3.285 $&$ 6.743$\\
			\hline
			10000 &$ 4.135 $&$ 7.384$\\
			\hline
				
			\end{tabular}
		\end{threeparttable}
	\end{center}
\end{table}

На рисунке~\ref{img:measure3} приведен график зависимости времени выполнения реализации от числа потоков. На рисунке~\ref{img:measure2} приведен график зависимости времени выполнения реализаций от размеров корпуса.  

\includeimage
{measure3} % Имя файла без расширения (файл должен быть расположен в директории inc/img/)
{f} % Обтекание (без обтекания)
{H} % Положение рисунка (см. figure из пакета float)
{1\textwidth} % Ширина рисунка
{График зависимости времени выполнения реализации от числа потоков} % Подпись рисунка


\includeimage
{measure2} % Имя файла без расширения (файл должен быть расположен в директории inc/img/)
{f} % Обтекание (без обтекания)
{H} % Положение рисунка (см. figure из пакета float)
{1\textwidth} % Ширина рисунка
{График зависимости времени выполнения реализаций от размеров корпуса} % Подпись рисунка


\section*{Вывод}

В результате анализа таблицы \ref{tbl:time_measurements}, было получено, что время выполнения многопоточной реализации алгоритма меньше времени выполнения однопоточной реализации при количестве потоков не больше 6, т.~е. при количестве потоков не больше количества физических ядер. При двух дополнительных потоках многопоточная реализации быстрее однопоточной в 2.28 раза.

Из таблицы \ref{tbl:time_measurements2}, можно сделать выводы, что при увеличении числа слов в корпусе время получения результата с использованием нескольких
потоков также увеличивается. 
При увеличении размеров корпуса с
1000 до 10000, время получения результата без использования дополнительных потоков
увеличилось в 10 раз, а время получения результата с использованием одного потока увеличилось в 27 раз. 
При этом при 10000 слов в корпусе реализация с одним дополнительным потоком быстрее реализации без дополнительных потоков в 1.8 раза. 
Таким образом, рекомендуется использование
двух вспомогательных потоков, т.~к. это позволит осуществлять поиск по корпусу с минимальными тратами на переключения контекста.