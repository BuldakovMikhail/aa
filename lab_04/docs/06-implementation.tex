\chapter{Технологический раздел}

В данном разделе будут приведены требования к программному обеспечению, средства реализации, листинг кода и функциональные тесты.


\section{Средства реализации}

Для реализации данной работы был выбран язык \textit{C++}~\cite{cpp}.
Данный выбор обусловлен следующим:
\begin{itemize}
	\item язык поддерживает все структуры данных, которые выбраны в результате проектирования;
	\item язык позволяет реализовать все алгоритмы, выбранные в результате проектирования;
	\item язык позволяет работать с нативными потоками~\cite{thread}. 
\end{itemize}

Время выполнения реализаций было замерено с помощью функции \textit{clock}~\cite{clock}. 
Для хранения слов использовалась структура данных \textit{wstring}~\cite{wstring}, в качестве массивов использовалась структура данных \textit{vector}~\cite{vector}.
В качестве примитива синхронизации использовался \textit{mutex}~\cite{mutex}.

\section{Сведения о модулях программы}

Данная программа разбита на следующие модули:
\begin{itemize}
	\item $main.cpp$ --- файл, содержащий функцию $main$;
	\item $correcter.cpp$ --- файл, содержащий код реализаций всех алгоритмов исправления ошибок;
	\item $measure\_time.cpp$ --- файл, в котором содержатся функции для замера и вывода времени выполнения реализаций алгоритмов;
	\item $utils.cpp$ --- файл, в котором содержатся вспомогательные функции;
	\item $levenstein.cpp$ --- файл, в котором содержится реализация алгоритма поиска расстояния Левенштейна.
\end{itemize}

\section{Реализация алгоритмов}

В листинге \ref{lst:algomain.cpp} приведена реализация алгоритма исправления ошибок без дополнительных потоков. 
В листинге \ref{lst:algomainmt.cpp} приведена реализация алгоритма исправления ошибок с использованием дополнительных потоков.
В листинге \ref{lst:threadfunc.cpp} приведена реализация функции, которая выполняется потоком.

\clearpage
\includelistingpretty
{algomain.cpp} % Имя файла с расширением (файл должен быть расположен в директории inc/lst/)
{python} % Язык программирования (необязательный аргумент)
{Функция исправления ошибок} % Подпись листинга

\clearpage

\includelistingpretty
{algomainmt.cpp} % Имя файла с расширением (файл должен быть расположен в директории inc/lst/)
{python} % Язык программирования (необязательный аргумент)
{Функция многопоточного исправления ошибок} % Подпись листинга

\clearpage

\includelistingpretty
{threadfunc.cpp} % Имя файла с расширением (файл должен быть расположен в директории inc/lst/)
{python} % Язык программирования (необязательный аргумент)
{Функция, выполняющаяся в потоке} % Подпись листинга

\clearpage

\section{Функциональные тесты}

В таблице \ref{tbl:func_tests} приведены функциональные тесты для разработанных алгоритмов исправления ошибок. Для данных тестов максимальное количество ошибок равно двум и из массива выбираются 3 слова. Все тесты пройдены успешно. В таблице \ref{tbl:func_tests} пустое слово обозначается с помощью $\lambda$.

\begin{table}[ht]
	\small
	\begin{center}
		\begin{threeparttable}
			\caption{Функциональные тесты}
			\label{tbl:func_tests}
			\begin{tabular}{|c|c|c|}
				\hline
				\bfseries Корпус
				& \bfseries Слово
				& \bfseries Ожидаемый результат \\ 
				\hline
				[Мама, Мыла, Раму] & Мама & [Мама] \\
				\hline
				[Мама, Мыла, Раму]  & мамы & [мама] \\
				\hline
				[Мама, Мыла, Раму]  & мыма & [мама, мыла] \\
				\hline
				[Мама, Мыла, Раму]  & ахтунг & [~] \\
				\hline
				[~]  & ахтунг & [~] \\
				\hline
				[~]  & $\lambda$ & [~] \\
				\hline
				[Мама, Мыла, Раму]  & $\lambda$ & [~] \\
				\hline
			\end{tabular}	
		\end{threeparttable}	
	\end{center}
\end{table}


\section*{Вывод}
Были разработаны и протестированы спроектированные алгоритмы исправления ошибок.

