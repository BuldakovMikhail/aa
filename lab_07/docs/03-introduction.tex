\chapter*{ВВЕДЕНИЕ}
\addcontentsline{toc}{chapter}{ВВЕДЕНИЕ}

Проблема поиска информации является одной из важнейших задач информатики.
Компьютерные методы информационного поиска --- активно развивающаяся, актуальная в научном и практическом аспекте тема современных публикаций. 
Развитие компьютерной техники влечет существенный рост объема информации, представляемой в электронном виде, влияние этого процесса на современные информационные технологии, включая поиск, отмечается в большинстве публикаций в периодических изданиях~\cite{intro}.



Цель данной лабораторной работы --- описать алгоритмы поиска подстроки в строке.
Для достижения поставленной цели необходимо выполнить следующие задачи:
\begin{itemize}
	\item описать стандартный алгоритм и алгоритм Кнута---Морриса---Пратта;
	\item разработать программное обеспечение, реализующее алгоритмы поиска подстроки в строке;
	\item выбрать инструменты для реализации алгоритмов и замера процессорного времени выполнения реализаций алгоритмов;
	\item проанализировать затраты реализаций алгоритмов по времени и количество сравнений в реализованных алгоритмах.
\end{itemize}
