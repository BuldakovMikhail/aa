\chapter{Аналитический раздел}

В данном разделе будут описаны два алгоритма поиска подстрок: стандартный, Кнута---Морриса---Пратта.

\section{Стандартный алгоритм}
Стандартный алгоритм начинается со сравнения первого символа текста с первым символом подстроки. 
При совпадении символов, сравниваются следующие символы. 
Так продолжается до тех пор, пока не окажется, что подстрока целиком совпала с отрезком текста, или пока не встретятся несовпадающие символы. 
В первом случае задача решена, во втором указатель текущего положения в тексте сдвигается на один символ и заново начинается сравнение с подстрокой~\cite{aa}.

\section{Алгоритм Кнута---Морриса---Пратта}
Алгоритм Кнута---Морриса---Пратта основан на принципе конечного автомата, однако он использует более простой метод обработки неподходящих символов.
В этом алгоритме состояния помечаются символами, совпадение с которыми должно в данный момент произойти. 
Из каждого состояния имеется два перехода: один соответствует успешному сравнению, другой --- несовпадению. 
При успешном сравнении осуществляется переход в следующий узел автомата, а в случае несовпадения осуществляется переход в предыдущий узел, отвечающий образцу. 
В программной реализации этого алгоритма применяется массив сдвигов, который создается для каждой подстроки, которая ищется в тексте.
Для каждого символа из подстроки рассчитывается значение, равное максимальной длине совпадающего префикса и суффикса относительно конкретного элемента подстроки. 
Создание этого массива позволяет при несовпадении строки сдвигать ее на расстояние большее, чем 1.


\section*{Вывод}

В данном разделе были описаны два алгоритма поиска подстрок: стандартный, Кнута---Морриса---Пратта.
