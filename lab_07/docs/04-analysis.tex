\chapter{Аналитический раздел}

В данном разделе будут описаны два алгоритма поиска подстрок: стандартный, Кнута---Морриса---Пратта.

\section{Стандартный алгоритм}
Стандартный алгоритм начинает со сравнения первого символа текста с первым символом подстроки. Если они совпадают, то происходит переход ко второму символу текста и подстроки. При совпадении сравниваются следующие символы. Так продолжается до тех пор, пока не окажется, что подстрока целиком совпала с отрезком текста, или пока не встретятся несовпадающие символы. В первом случае задача решена, во втором мы сдвигаем указатель текущего положения в тексте на один символ и заново начинаем сравнение с подстрокой~\cite{aa}.

\section{Алгоритм Кнута---Морриса---Пратта}
Алгоритм Кнута---Морриса---Пратта основан на принципе конечного автомата, однако он использует более простой метод обработки неподходящих символов.
В этом алгоритме состояния помечаются символами, совпадение с которыми должно в данный момент произойти. 
Из каждого состояния имеется два перехода: один соответствует успешному сравнению, другой --- несовпадению. 
Успешное сравнение переводит нас в следующий узел автомата, а в случае несовпадения мы попадаем в предыдущий узел, отвечающий образцу. 
В программной реализации этого алгоритма применяется массив сдвигов, который создается для каждой подстроки, которая ищется в тексте.
Для каждого символа из подстроки рассчитывается значение, равное максимальной длине совпадающего префикса и суффикса относительно конкретного элемента подстроки. Создание этого массива позволяет при несовпадении строки сдвигать ее на расстояние, большее, чем 1 (в отличие от стандартного алгоритма).


\section*{Вывод}

В данном разделе были описаны два алгоритма поиска подстрок: стандартный, Кнута---Морриса---Пратта.
