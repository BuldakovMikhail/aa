\chapter{Исследовательский раздел}

В данном разделе будут приведены: пример работы программы, постановка эксперимента и сравнительный анализ алгоритмов на основе полученных данных.

\section{Демонстрация работы программы}


На рисунке \ref{img:program} представлена демонстрация работы разработанного программного обеспечения, а именно показаны результаты поиска подстроки <<ам>> в строке <<мама>>.  
\clearpage

\includeimage
{program} % Имя файла без расширения (файл должен быть расположен в директории inc/img/)
{f} % Обтекание (без обтекания)
{h} % Положение рисунка (см. figure из пакета float)
{1\textwidth} % Ширина рисунка
{Демонстрация работы программы при поиске подстрок} % Подпись рисунка

\clearpage


\section{Технические характеристики}

Технические характеристики устройства, на котором выполнялись замеры по времени, следующие:
\begin{itemize}
	\item процессор: AMD Ryzen 5 4600H 3 ГГц \cite{amd};
	\item оперативная память: 16 ГБайт;
	\item операционная система: Windows 10 Pro 64-разрядная система версии 22H2 \cite{windows}.
\end{itemize}

При замерах времени ноутбук был включен в сеть электропитания и был нагружен только системными приложениями.

\section{Время выполнения реализаций алгоритмов}

Результаты замеров времени выполнения реализаций алгоритмов поиска приведены в таблицах \ref{tbl:time_measurements} -- \ref{tbl:time_measurements_2}.
Замеры времени проводились на строках одной длины и усреднялись для каждого набора одинаковых экспериментов.

В таблицах \ref{tbl:time_measurements} -- \ref{tbl:time_measurements_2} используются следующие обозначения: 
\begin{itemize}
	\item С --- реализация стандартного алгоритма;
	\item КМП --- реализация алгоритма Кнута---Морриса---Пратта.
\end{itemize}

\begin{table}[h]
	\begin{center}
		\begin{threeparttable}
			\captionsetup{justification=raggedright,singlelinecheck=off}
			\caption{Время работы реализаций алгоритмов на строках, где искомая подстрока стоит на первой позиции (в с)}
			\label{tbl:time_measurements}
			\begin{tabular}{|c|c|c|}
				\hline
				Длина строки &  КМП  & С \\
				\hline
				256 &$ 5.656\cdot 10^{-6} $&$ 1.750\cdot 10^{-6}$\\
				\hline
				512 &$ 5.406\cdot 10^{-6} $&$ 1.594\cdot 10^{-6}$\\
				\hline
				1024 &$ 5.406\cdot 10^{-6} $&$ 1.500\cdot 10^{-6}$\\
				\hline
				2048 &$ 5.438\cdot 10^{-6} $&$ 1.438\cdot 10^{-6}$\\
				\hline
				4096 &$ 5.375\cdot 10^{-6} $&$ 1.438\cdot 10^{-6}$\\
				\hline
				8192 &$ 5.438\cdot 10^{-6} $&$ 1.438\cdot 10^{-6}$\\
				\hline
				16384 &$ 5.313\cdot 10^{-6} $&$ 1.469\cdot 10^{-6}$\\
				\hline
				32768 &$ 5.406\cdot 10^{-6} $&$ 1.406\cdot 10^{-6}$\\
				\hline
				65536 &$ 5.344\cdot 10^{-6} $&$ 1.438\cdot 10^{-6}$\\
				\hline
			\end{tabular}
		\end{threeparttable}
	\end{center}
\end{table}

\begin{table}[h]
	\begin{center}
		\begin{threeparttable}
			\captionsetup{justification=raggedright,singlelinecheck=off}
			\caption{Время работы реализации алгоритмов на строках, где искомая подстрока стоит в конце (в с)}
			\label{tbl:time_measurements_1}
			\begin{tabular}{|c|c|c|}
				\hline
				Длина строки &  КМП  & С \\
				\hline
				256 &$ 1.250\cdot 10^{-4} $&$ 3.125\cdot 10^{-4}$\\
				\hline
				512 &$ 3.125\cdot 10^{-4} $&$ 6.875\cdot 10^{-4}$\\
				\hline
				1024 &$ 5.625\cdot 10^{-4} $&$ 1.437\cdot 10^{-3}$\\
				\hline
				2048 &$ 1.250\cdot 10^{-3} $&$ 2.875\cdot 10^{-3}$\\
				\hline
				4096 &$ 2.375\cdot 10^{-3} $&$ 5.875\cdot 10^{-3}$\\
				\hline
				8192 &$ 4.562\cdot 10^{-3} $&$ 9.938\cdot 10^{-3}$\\
				\hline
				16384 &$ 8.250\cdot 10^{-3} $&$ 2.013\cdot 10^{-2}$\\
				\hline
				32768 &$ 1.619\cdot 10^{-2} $&$ 4.069\cdot 10^{-2}$\\
				\hline
				65536 &$ 3.331\cdot 10^{-2} $&$ 8.125\cdot 10^{-2}$\\
				\hline
			\end{tabular}
		\end{threeparttable}
	\end{center}
\end{table}

\begin{table}[h]
	\begin{center}
		\begin{threeparttable}
			\captionsetup{justification=raggedright,singlelinecheck=off}
			\caption{Время работы реализации алгоритмов  на строках, где искомая подстрока отсутствует (в с)}
			\label{tbl:time_measurements_2}
			\begin{tabular}{|c|c|c|}
				\hline
				Длина строки &  КМП  & С \\
				\hline
				256 &$ 1.875\cdot 10^{-4} $&$ 3.125\cdot 10^{-4}$\\
				\hline
				512 &$ 3.125\cdot 10^{-4} $&$ 7.500\cdot 10^{-4}$\\
				\hline
				1024 &$ 5.625\cdot 10^{-4} $&$ 1.313\cdot 10^{-3}$\\
				\hline
				2048 &$ 1.250\cdot 10^{-3} $&$ 2.938\cdot 10^{-3}$\\
				\hline
				4096 &$ 2.188\cdot 10^{-3} $&$ 5.000\cdot 10^{-3}$\\
				\hline
				8192 &$ 4.125\cdot 10^{-3} $&$ 1.000\cdot 10^{-2}$\\
				\hline
				16384 &$ 8.500\cdot 10^{-3} $&$ 2.025\cdot 10^{-2}$\\
				\hline
				32768 &$ 1.625\cdot 10^{-2} $&$ 4.088\cdot 10^{-2}$\\
				\hline
				65536 &$ 3.275\cdot 10^{-2} $&$ 7.994\cdot 10^{-2}$\\
				\hline
				
			\end{tabular}
		\end{threeparttable}
	\end{center}
\end{table}

\clearpage
На рисунках \ref{img:best} -- \ref{img:without} изображены графики зависимостей времени выполнения реализаций алгоритмов поиска от размеров строки.

\includeimage
{best} % Имя файла без расширения (файл должен быть расположен в директории inc/img/)
{f} % Обтекание (без обтекания)
{h} % Положение рисунка (см. figure из пакета float)
{1\textwidth} % Ширина рисунка
{Сравнение реализаций алгоритмов по времени выполнения на массивах, отсортированных в обратном порядке} % Подпись рисунка

\includeimage
{worst} % Имя файла без расширения (файл должен быть расположен в директории inc/img/)
{f} % Обтекание (без обтекания)
{h} % Положение рисунка (см. figure из пакета float)
{1\textwidth} % Ширина рисунка
{Сравнение реализаций алгоритмов по времени выполнения на отсортированных массивах} % Подпись рисунка

\includeimage
{without} % Имя файла без расширения (файл должен быть расположен в директории inc/img/)
{f} % Обтекание (без обтекания)
{h} % Положение рисунка (см. figure из пакета float)
{1\textwidth} % Ширина рисунка
{Сравнение реализаций алгоритмов по времени выполнения на случайно упорядоченных массивах} % Подпись рисунка

\clearpage


\section{Количество сравнений}

Подсчет количества сравнений производился для строки длинной 256 символов, без учета длины подстроки, состоящей из символов <<а>>. Подстрокой являлась строка <<аааааффффф>>. 

Результаты подсчетов числа сравнений представлены на диаграммах~\ref{img:standart} и \ref{img:kmp}.

\includeimage
{standart} % Имя файла без расширения (файл должен быть расположен в директории inc/img/)
{f} % Обтекание (без обтекания)
{h} % Положение рисунка (см. figure из пакета float)
{1\textwidth} % Ширина рисунка
{Сравнение реализаций алгоритмов по времени выполнения на случайно упорядоченных массивах} % Подпись рисунка


\includeimage
{kmp} % Имя файла без расширения (файл должен быть расположен в директории inc/img/)
{f} % Обтекание (без обтекания)
{h} % Положение рисунка (см. figure из пакета float)
{1\textwidth} % Ширина рисунка
{Сравнение реализаций алгоритмов по времени выполнения на случайно упорядоченных массивах} % Подпись рисунка


\section*{Вывод}
