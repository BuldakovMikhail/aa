\chapter*{ЗАКЛЮЧЕНИЕ}
\addcontentsline{toc}{chapter}{ЗАКЛЮЧЕНИЕ}

Цель данной лабораторной работы была достигнута, а именно были описаны алгоритмы поиска подстроки в строке.

Для достижения поставленной цели были выполнены следующие задачи:
\begin{itemize}
	\item описаны стандартный алгоритм и алгоритм Кнута---Морриса---Пратта;
	\item разработано программное обеспечение, реализующее алгоритмы поиска подстроки в строке;
	\item выбраны инструменты для реализации алгоритмов и замера процессорного времени выполнения реализаций алгоритмов;
	\item проанализированы затраты реализаций алгоритмов по времени и количество сравнений в реализованных алгоритмах.
\end{itemize}

В результате замеров времени выполнения реализаций алгоритмов поиска было выявлено, что реализация алгоритма Кнута---Морисса---Пратта медленнее реализации стандартного алгоритма в 3 раза, в случае когда подстрока стоит в начале, при длине подстроки 10 и размеров строки 256. 
Такой результат обусловлен необходимостью вычислять массив сдвигов для подстроки в реализации алгоритма Кнута---Морисса---Пратта.
В то же время, реализация алгоритма Кнута---Морисса---Пратта выигрывает по скорости реализацию стандартного алгоритма в 2.4 раза при длине строк 65536, когда подстрока отсутствует или стоит в конце строки. 
Такой результат обусловлен тем, что в реализации алгоритма Кнута---Морисса---Пратта используются особенности искомой подстроки, что позволяет осуществлять смещения сразу на несколько шагов.

В результате подсчетов количества сравнений было выявлено, что для реализации алгоритма Кнута---Морриса---Пратта худшим является случай, когда подстрока отсутствует в строке. 
Для реализации стандартного алгоритма, худшим является случай, когда подстрока стоит на последних позициях в строке. 
При этом в реализации алгоритма Кнута---Морриса---Пратта осуществляется примерно в 3 раза меньше сравнений при индексе подстроки 256, чем в реализации стандартного алгоритма.


