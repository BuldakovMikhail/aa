\chapter{Конструкторский раздел}

В данной части работы будут рассмотрены псевдокод стандартного алгоритма и псевдокод алгоритма Кнута---Морриса---Пратта.


\section{Требования к программному обеспечению}

Программа должна поддерживать два режима работы: режим массового замера времени и режим поиска подстроки в заданной строке.

Режим массового замера времени должен обладать следующей функциональностью:
\begin{itemize}
	\item генерировать строки различного размер для проведения замеров;
	\item осуществлять массовый замер времени, используя сгенерированные данные;
	\item результаты замеров должны быть представлены в виде таблицы и графика.
\end{itemize}	

К режиму поиска подстроки выдвигается следующий ряд требований:
\begin{itemize}
	\item возможность работать со строками разного размера, которые вводит пользователь;
	\item возможность правильно обрабатывать кириллицу;
	\item наличие интерфейса для выбора действий;
	\item на выходе программы, индекс подстроки в строке или $-1$, если подстрока не найдена.
\end{itemize}

\section{Разработка алгоритмов}

В листингах~\ref{alg:native}--\ref{alg:kmp} рассмотрены псевдокоды алгоритмов поиска, входными данными для них являются:
\begin{itemize}
	\item $s$~--- строка, в которой осуществляется поиск;
	\item $substr$~--- подстрока, которую требуется найти.
\end{itemize}

В случае отсутствия подстроки в строке происходит возврат $-1$~--- невалидного индекса в строке. 
Оператор $\gets$~обозначает присваивание значения переменной, оператор $[i]$ обозначает получение буквы из строки с индексом $i$, оператор $len(str)$ обозначает длину строки $str$, иные операторы подобны математическим.

\begin{algorithm}[H]
	\caption{Псевдокод стандартного алгоритма.}\label{alg:native}
	\begin{algorithmic}
		\For {от $i=0$ до $len(s) - len(substr)$}
		\State {$flag \gets 0$}
		\For {от $j=0$ до $len(substr) - 1$}
		\If {$s[i + j] \ne substr[j]$}
		\State {$flag \gets 1$}
		\State {Выйти из цикла}
		\EndIf
		\EndFor
		\If {$flag = 0$}
		\State \Return {$i$}
		\EndIf
		\EndFor
		\State	\Return {$-1$}
	\end{algorithmic}
\end{algorithm}

\begin{algorithm}[H]
	\caption{Псевдокод алгоритма Кнута---Морриса---Пратта.}\label{alg:kmp}
	\begin{algorithmic}
		\State{Создать массив целых чисел $next$ длины $len(substr) + 1$}
		\State{Заполнить массив $next$ нулями}
		\For {от $i=1$ до $len(substr) - 1$}
			\State {$j \gets next[i]$}
			\While{$j > 0$ и $substr[j] \ne substr[i]$}
				\State {$j \gets next[j]$}
			\EndWhile
			\If {$j > 0$ или $substr[j] = substr[i]$}
				\State {$next[i + 1] \gets j + 1$}
			\EndIf
		\EndFor
		\State {$i \gets 0$}
		\State {$j \gets 0$}
		\While{$i < len(s)$}
		\If{$j < len(substr)$ и $s[i] = substr[j]$}
			\State{$j \gets j + 1$}
			\If{$j = len(substr)$}
				\State \Return {$i - j + 1$}
			\EndIf
		\ElsIf{$j > 0$}
			\State{$j \gets next[j]$}
			\State{$i \gets i - 1$}
		\EndIf
		\State{$i \gets i + 1$}
		\EndWhile
		\State \Return {$-1$}
		
		
	\end{algorithmic}
\end{algorithm}

\section*{Вывод}
В данной части работы был написан псевдокод для алгоритмов стандартного поиска и алгоритма Кнута---Морриса---Пратта.