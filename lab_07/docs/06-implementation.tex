\chapter{Технологический раздел}

В данном разделе будут приведены средства реализации, листинг кода и функциональные тесты.


\section{Средства реализации}

Для реализации данной работы был выбран язык \textit{Python}~\cite{python}.
Данный выбор обусловлен следующим:
\begin{itemize}
	\item язык поддерживает все структуры данных, которые выбраны в результате проектирования;
	\item язык позволяет реализовать все алгоритмы, выбранные в результате проектирования;
	\item язык позволяет замерять процессорное время с помощью модуля \textit{time}. 
\end{itemize}

Процессорное время было замерено с помощью функции \textit{process\_time()} из модуля \textit{time}~\cite{python-time}.

\section{Сведения о модулях программы}

Данная программа разбита на следующие модули:
\begin{itemize}
	\item $main.py$ --- файл, содержащий функцию $main$;
	\item $algorithms.py$ --- файл, содержащий код реализаций всех алгоритмов поиска;
	\item $compare\_time.py$ --- файл, в котором содержатся функции для замера и вывода времени выполнения реализаций алгоритмов;
	\item $count\_compares.py$ --- файл, в котором содержатся функции для подсчета количества сравнений в реализациях алгоритмов.
\end{itemize}

\section{Реализации алгоритмов}

В листингах \ref{lst:native.py} и \ref{lst:kmp.py} приведены реализации алгоритмов стандартного поиска и алгоритма Кнута---Морриса---Пратта соответственно.

\clearpage
\includelistingpretty
{native.py} % Имя файла с расширением (файл должен быть расположен в директории inc/lst/)
{python} % Язык программирования (необязательный аргумент)
{Функция стандартного поиска} % Подпись листинга

\clearpage

\includelistingpretty
{kmp.py} % Имя файла с расширением (файл должен быть расположен в директории inc/lst/)
{python} % Язык программирования (необязательный аргумент)
{Функция алгоритма Кнута---Морриса---Пратта} % Подпись листинга

\clearpage

\section{Функциональные тесты}

В таблице \ref{tbl:func_tests} приведены функциональные тесты для разработанных алгоритмов поиска. Пустая строка обозначается с помощью символа $\lambda$. 
Все тесты пройдены успешно.

\begin{table}[ht]
	\small
	\begin{center}
		\begin{threeparttable}
			\caption{Функциональные тесты}
			\label{tbl:func_tests}
			\begin{tabular}{|c|c|c|c|}
				\hline
				\bfseries Строка
				& \bfseries Подстрока
				& \bfseries Ожидаемый результат
				& \bfseries Фактический результат \\ 
				\hline
				мама & ам & 1 & 1\\
				\hline
				абоба & оба & 2 & 2\\
				\hline
				абабабцб & абабцб & 2 & 2\\
				\hline
				мама & пап & -1 & -1\\
				\hline
				абабаба & аб & 0 & 0\\
				\hline
				мамы & ы & 3 & 3\\
				\hline
				ы & ы & 0 & 0\\
				\hline
				$\lambda$ & ы & -1 & -1\\
				\hline
				ы & $\lambda$ & -1 & -1\\
				\hline
				$\lambda$ & $\lambda$ & -1 & -1\\
				\hline
				абоба & абобаабоба & -1 & -1\\
				\hline
			\end{tabular}	
		\end{threeparttable}	
	\end{center}
\end{table}


\section*{Вывод}
Были разработаны и протестированы спроектированные алгоритмы поиска.