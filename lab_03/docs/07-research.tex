\chapter{Исследовательский раздел}

В данном разделе будут приведены: пример работы программы, постановка эксперимента и сравнительный анализ алгоритмов на основе полученных данных.

\section{Демонстрация работы программы}


На рисунке \ref{img:program} представлена демонстрация работы разработанного программного обеспечения, а именно показаны результаты сортировки массива $[1, 6, 3, 2, 1, 3, 4]$.  
\clearpage

\includeimage
{program} % Имя файла без расширения (файл должен быть расположен в директории inc/img/)
{f} % Обтекание (без обтекания)
{h} % Положение рисунка (см. figure из пакета float)
{1\textwidth} % Ширина рисунка
{Демонстрация работы программы при сортировке массива} % Подпись рисунка

\clearpage


\section{Технические характеристики}

Технические характеристики устройства, на котором выполнялись замеры по времени, следующие:
\begin{itemize}
	\item процессор: AMD Ryzen 5 4600H 3 ГГц \cite{amd};
	\item оперативная память: 16 ГБайт;
	\item операционная система: Windows 10 Pro 64-разрядная система версии 22H2 \cite{windows}.
\end{itemize}

При замерах времени ноутбук был включен в сеть электропитания и был нагружен только системными приложениями.

\section{Время выполнения реализаций алгоритмов}

Результаты замеров времени выполнения реализаций алгоритмов сортировок приведены в таблицах \ref{tbl:time_measurements} -- \ref{tbl:time_measurements_rand}.
Замеры времени проводились на массивах одного размера и усреднялись для каждого набора одинаковых экспериментов.

В таблицах \ref{tbl:time_measurements} -- \ref{tbl:time_measurements_rand} используются следующие обозначения: 
\begin{itemize}
	\item Ш --- реализация алгоритма сортировки Шелла;
	\item Г --- реализация алгоритма гномьей сортировки;
	\item П --- реализация алгоритма пирамидальной сортировки.
\end{itemize}

\begin{table}[h]
	\begin{center}
		\begin{threeparttable}
			\captionsetup{justification=raggedright,singlelinecheck=off}
			\caption{Время работы реализации алгоритмов на массивах, отсортированных в обратном порядке (в с)}
			\label{tbl:time_measurements}
			\begin{tabular}{|c|c|c|c|}
				\hline
				Размер массива &  Ш  & Г & П \\
				\hline
				1000 &$ 5.937\cdot 10^{-4} $&$ 3.125\cdot 10^{-4} $&$ 3.031\cdot 10^{-3}$\\
				\hline
				2000 &$ 1.313\cdot 10^{-3} $&$ 1.187\cdot 10^{-3} $&$ 6.781\cdot 10^{-3}$\\
				\hline
				3000 &$ 2.281\cdot 10^{-3} $&$ 2.594\cdot 10^{-3} $&$ 1.109\cdot 10^{-2}$\\
				\hline
				4000 &$ 3.125\cdot 10^{-3} $&$ 4.500\cdot 10^{-3} $&$ 1.578\cdot 10^{-2}$\\
				\hline
				5000 &$ 4.281\cdot 10^{-3} $&$ 6.938\cdot 10^{-3} $&$ 1.975\cdot 10^{-2}$\\
				\hline
				6000 &$ 5.281\cdot 10^{-3} $&$ 9.938\cdot 10^{-3} $&$ 2.409\cdot 10^{-2}$\\
				\hline
				7000 &$ 6.094\cdot 10^{-3} $&$ 1.341\cdot 10^{-2} $&$ 2.859\cdot 10^{-2}$\\
				\hline
				8000 &$ 6.875\cdot 10^{-3} $&$ 1.734\cdot 10^{-2} $&$ 3.347\cdot 10^{-2}$\\
				\hline
				9000 &$ 8.531\cdot 10^{-3} $&$ 2.234\cdot 10^{-2} $&$ 3.806\cdot 10^{-2}$\\
				\hline
			\end{tabular}
		\end{threeparttable}
	\end{center}
\end{table}

\begin{table}[h]
	\begin{center}
		\begin{threeparttable}
			\captionsetup{justification=raggedright,singlelinecheck=off}
			\caption{Время работы реализации алгоритмов на отсортированных массивах (в с)}
			\label{tbl:time_measurements_sorted}
			\begin{tabular}{|c|c|c|c|}
				\hline
				Размер массива &  Ш  & Г & П \\
				\hline
				1000 &$ 5.937\cdot 10^{-4} $&$ 9.375\cdot 10^{-5} $&$ 3.063\cdot 10^{-3}$\\
				\hline
				2000 &$ 1.500\cdot 10^{-3} $&$ 2.188\cdot 10^{-4} $&$ 6.938\cdot 10^{-3}$\\
				\hline
				3000 &$ 2.250\cdot 10^{-3} $&$ 2.812\cdot 10^{-4} $&$ 1.109\cdot 10^{-2}$\\
				\hline
				4000 &$ 3.187\cdot 10^{-3} $&$ 4.062\cdot 10^{-4} $&$ 1.528\cdot 10^{-2}$\\
				\hline
				5000 &$ 4.313\cdot 10^{-3} $&$ 5.000\cdot 10^{-4} $&$ 1.984\cdot 10^{-2}$\\
				\hline
				6000 &$ 5.125\cdot 10^{-3} $&$ 5.625\cdot 10^{-4} $&$ 2.409\cdot 10^{-2}$\\
				\hline
				7000 &$ 6.031\cdot 10^{-3} $&$ 6.563\cdot 10^{-4} $&$ 2.853\cdot 10^{-2}$\\
				\hline
				8000 &$ 6.594\cdot 10^{-3} $&$ 8.125\cdot 10^{-4} $&$ 3.369\cdot 10^{-2}$\\
				\hline
				9000 &$ 8.500\cdot 10^{-3} $&$ 9.063\cdot 10^{-4} $&$ 3.825\cdot 10^{-2}$\\
				\hline		
			\end{tabular}
		\end{threeparttable}
	\end{center}
\end{table}

\begin{table}[h]
	\begin{center}
		\begin{threeparttable}
			\captionsetup{justification=raggedright,singlelinecheck=off}
			\caption{Время работы реализации алгоритмов на случайно упорядоченных массивах (в с)}
			\label{tbl:time_measurements_rand}
			\begin{tabular}{|c|c|c|c|}
				\hline
				Размер массива &  Ш  & Г & П \\
				\hline
				1000 &$ 1.969\cdot 10^{-3} $&$ 4.687\cdot 10^{-4} $&$ 8.484\cdot 10^{-3}$\\
				\hline
				2000 &$ 4.484\cdot 10^{-3} $&$ 1.359\cdot 10^{-3} $&$ 1.891\cdot 10^{-2}$\\
				\hline
				3000 &$ 7.547\cdot 10^{-3} $&$ 2.687\cdot 10^{-3} $&$ 3.008\cdot 10^{-2}$\\
				\hline
				4000 &$ 1.003\cdot 10^{-2} $&$ 4.469\cdot 10^{-3} $&$ 4.127\cdot 10^{-2}$\\
				\hline
				5000 &$ 1.358\cdot 10^{-2} $&$ 6.734\cdot 10^{-3} $&$ 5.317\cdot 10^{-2}$\\
				\hline
				6000 &$ 1.648\cdot 10^{-2} $&$ 9.312\cdot 10^{-3} $&$ 6.509\cdot 10^{-2}$\\
				\hline
				7000 &$ 1.914\cdot 10^{-2} $&$ 1.198\cdot 10^{-2} $&$ 7.742\cdot 10^{-2}$\\
				\hline
				8000 &$ 2.178\cdot 10^{-2} $&$ 1.555\cdot 10^{-2} $&$ 8.947\cdot 10^{-2}$\\
				\hline
				9000 &$ 2.667\cdot 10^{-2} $&$ 1.956\cdot 10^{-2} $&$ 1.035\cdot 10^{-1}$\\
				\hline
			\end{tabular}
		\end{threeparttable}
	\end{center}
\end{table}

\clearpage
На рисунках \ref{img:reverseSorted} -- \ref{img:randSorted} изображены графики зависимостей времени выполнения реализаций сортировок от размеров массивов.

\includeimage
{reverseSorted} % Имя файла без расширения (файл должен быть расположен в директории inc/img/)
{f} % Обтекание (без обтекания)
{h} % Положение рисунка (см. figure из пакета float)
{1\textwidth} % Ширина рисунка
{Сравнение реализаций алгоритмов по времени выполнения на массивах, отсортированных в обратном порядке} % Подпись рисунка

\includeimage
{sorted} % Имя файла без расширения (файл должен быть расположен в директории inc/img/)
{f} % Обтекание (без обтекания)
{h} % Положение рисунка (см. figure из пакета float)
{1\textwidth} % Ширина рисунка
{Сравнение реализаций алгоритмов по времени выполнения на отсортированных массивах} % Подпись рисунка

\includeimage
{randSorted} % Имя файла без расширения (файл должен быть расположен в директории inc/img/)
{f} % Обтекание (без обтекания)
{h} % Положение рисунка (см. figure из пакета float)
{1\textwidth} % Ширина рисунка
{Сравнение реализаций алгоритмов по времени выполнения на случайно упорядоченных массивах} % Подпись рисунка

\clearpage


\section{Характеристики по памяти}

Введем следующие обозначения:
\begin{itemize}
	\item $n$ --- длина массива, который необходимо отсортировать $arr$;
	\item $size()$ --- функция, вычисляющая размер в байтах;
	\item $int$ --- целочисленный тип данных;
	\item $float$ --- вещественный тип данных.
\end{itemize}

Максимальное требование по памяти реализации алгоритма Шелла складывается из 5 локальных переменных типа $int$, адреса возврата $int$, возвращаемого значения (ссылки) и рассчитывается по формуле \eqref{mem:shell}.
\begin{equation}
	\label{mem:shell}
	f_{memshell} = 6 \cdot size(int) + size(float *).
\end{equation}

Аналогично, реализация алгоритма гномьей сортировки максимально требует памяти под 2 локальные переменные типа $int$, адрес возврата $int$, возвращаемое значение (ссылку), т.~о. требования по памяти рассчитываются по формуле \eqref{mem:gnome}.
\begin{equation}
	\label{mem:gnome}
	f_{memgnome} = 3 \cdot size(int) + size(float *).
\end{equation}

Максимальное требование реализации алгоритма пирамидальной сортировки по памяти формируется из 3 локальных переменных типа $int$, адреса возврата $int$, возвращаемого по ссылке значения, при этом максимальная глубина рекурсии подпрограммы \textit{heapify} равна $\log_2 N$. 
В подпрограмме \textit{heapify} используются 3 локальных переменных типа $int$, а для вызова в нее необходимо передать массив по ссылке и 2 переменные типа $int$. 
Память требуемая реализацией алгоритма пирамидальной сортировки рассчитывается по формуле \eqref{mem:heap}.
\begin{equation}
	\label{mem:heap}
	f_{heap} = 3 \cdot size(int) + size(float *) + \log_2 N (6 \cdot size(int) + size(float *)).
\end{equation}

\section*{Вывод}

В результате замеров времени выполнения реализаций различных алгоритмов было выявлено, что для массивов длины 9000, отсортированных в обратном порядке, реализация алгоритма Шелла по времени оказалась в 2.6 раза лучше, чем реализация гномьей сортировки, и в 4.5 раза лучше реализации пирамидальной сортировки. 
В свою очередь, реализация гномьей сортировки оказалась лучше в 1.7 раз по времени выполнения, чем реализация пирамидальной сортировки.
Что соответствует теоретической оценке трудоемкости. 
Поскольку алгоритм Шелла по теоретической оценке трудоемкости в худшем случае выигрывает по константе гномью сортировку, $O(\frac{32}{3}N^2)$ и $O(\frac{23}{2}N^2)$ соответственно, при этом алгоритм пирамидальной сортировки, обладая теоретической оценкой трудоемкости $O(\frac{87}{2} N \log_2N)$, из-за большой константы проигрывает остальным, хоть и обладает меньшей скоростью роста.

Для отсортированных массивов длинной 9000 реализация гномьей сортировки оказалась лучше по времени в 9 раза, чем реализация алгоритма Шелла, и в 42 раза лучше, чем реализация пирамидальной сортировки. 
В свою очередь, реализация пирамидальной сортировки на отсортированных массивах, оказалась хуже в 4.5 раза, чем реализации алгоритма Шелла по времени выполнения. 
Что соответствует теоретической оценке трудоемкости.
Поскольку в лучшем случае алгоритм гномьей сортировки обладает наименьшей асимптотической оценкой и малой константой $O(7N)$. А алгоритм Шелла выигрывает алгоритм пирамидальной сортировки по константе, $O(10 N \log_2N)$ и $O(29 N \log_2N)$ соответственно.

Для случайно упорядоченных массивов длинной 9000 реализация гномьей сортировки оказалась лучше по времени в 1.4 раза, чем реализация алгоритма Шелла, и в 5.3 раза лучше, чем реализация пирамидальной сортировки. 
В свою очередь, реализация пирамидальной сортировки на случайно упорядоченных массивах, оказалась хуже в 3.9 раз, чем реализации алгоритма Шелла по времени выполнения.

При этом меньше всего памяти требует реализация гномьей сортировки, а больше всего --- реализация пирамидальной сортировки.

Стоит заметить, что для обратно упорядоченных массивов длинной менее 2500, реализация гномьей сортировки показывала лучшие результаты. 
На случайно упорядоченных массивах, гномья сортировка хоть и оказалась эффективней, но обладая большей скоростью роста, при больших длинах массивов она окажется менее эффективной, чем сортировка Шелла и пирамидальная. 
То же касается и пирамидальной сортировки, за счет большой константы, она показала худший результат во всех случаях, но поскольку асимптотическая оценка этого алгоритма меньше остальных для случайно упорядоченных и обратно упорядоченных массивов, данная реализация покажет лучшую эффективность по времени при гораздо больших длинах массивов.