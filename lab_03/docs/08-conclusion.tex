\chapter*{ЗАКЛЮЧЕНИЕ}
\addcontentsline{toc}{chapter}{ЗАКЛЮЧЕНИЕ}

Цель данной лабораторной работы была достигнута, а именно были исследованы алгоритмы сортировок.


Для достижения поставленной цели были выполнены следующие задачи.
\begin{itemize}
	\item описаны следующие алгоритмы сортировок:
	\begin{itemize}
		\item гномья;
		\item пирамидальная;
		\item Шелла;
	\end{itemize}
	\item разработано программное обеспечение, реализующее алгоритмы сортировок;
	\item выбраны инструменты для реализации алгоритмов и замера процессорного времени их выполнения;
	\item проведен анализ затрат реализаций алгоритмов по времени. 
\end{itemize}

В результате исследования реализаций различных алгоритмов было получено, что для массивов длины 9000, отсортированных в обратном порядке, реализация алгоритма Шелла по времени оказалась в 2.6 раза лучше, чем реализация гномьей сортировки, и в 4.5 раза лучше реализации пирамидальной сортировки.
В свою очередь, реализация гномьей сортировки оказалась лучше в 1.7 раз по времени выполнения, чем реализация пирамидальной сортировки.
Для отсортированных массивов длинной 9000 реализация гномьей сортировки оказалась лучше по времени в 9 раза, чем реализация алгоритма Шелла, и в 42 раза лучше, чем реализация пирамидальной сортировки. 
В свою очередь, реализация пирамидальной сортировки на отсортированных массивах, оказалась хуже в 4.5 раза, чем реализации алгоритма Шелла по времени выполнения.
Для случайно упорядоченных массивов длинной 9000 реализация гномьей сортировки оказалась лучше по времени в 1.4 раза, чем реализация алгоритма Шелла, и в 5.3 раза лучше, чем реализация пирамидальной сортировки. 
В свою очередь, реализация пирамидальной сортировки на случайно упорядоченных массивах, оказалась хуже в 3.9 раз, чем реализации алгоритма Шелла по времени выполнения.

Для обратно упорядоченных массивов длинной менее 2500, реализация гномьей сортировки показывала лучшие результаты. 
На случайно упорядоченных массивах, гномья сортировка хоть и оказалась эффективней, но обладая большей скоростью роста, при больших длинах массивов она окажется менее эффективной, чем сортировка Шелла и пирамидальная. 
То же касается и пирамидальной сортировки, за счет большой константы, она показала худший результат во всех случаях, но поскольку асимптотическая оценка этого алгоритма меньше остальных для случайно упорядоченных и обратно упорядоченных массивов, данная реализация покажет лучшую эффективность по времени при гораздо больших длинах массивов.