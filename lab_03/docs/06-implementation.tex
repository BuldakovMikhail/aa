\chapter{Технологический раздел}

В данном разделе будут приведены требования к программному обеспечению, средства реализации, листинг кода и функциональные тесты.


\section{Средства реализации}

Для реализации данной работы был выбран язык \textit{Python}~\cite{python}.
Такой выбор обусловлен опытом работы с этим языком программирования.
Также данный язык позволяет замерять процессорное время с помощью модуля \textit{time}. 

Процессорное время было замерено с помощью функции \textit{process\_time()} из модуля \textit{time}~\cite{python-time}.

\section{Сведения о модулях программы}

Данная программа разбита на следующие модули:
\begin{itemize}
	\item $main.py$ --- файл, содержащий функцию $main$;
	\item $algorithms.py$ --- файл, содержащий код реализаций всех алгоритмов сортировок;
	\item $compare\_time.py$ --- файл, в котором содержатся функции для замера и вывода времени выполнения реализаций алгоритмов.
\end{itemize}

\section{Реализация алгоритмов}

В листингах \ref{lst:shell.py} -- \ref{lst:heapsort.py} приведены реализации сортировок Шелла, гномьей и пирамидальной соответственно.

\clearpage
\includelistingpretty
{shell.py} % Имя файла с расширением (файл должен быть расположен в директории inc/lst/)
{python} % Язык программирования (необязательный аргумент)
{Функция сортировки массива по алгоритму Шелла} % Подпись листинга

\clearpage

\includelistingpretty
{gnome.py} % Имя файла с расширением (файл должен быть расположен в директории inc/lst/)
{python} % Язык программирования (необязательный аргумент)
{Функция гномьей сортировки массива} % Подпись листинга

\clearpage

\includelistingpretty
{heapsort.py} % Имя файла с расширением (файл должен быть расположен в директории inc/lst/)
{python} % Язык программирования (необязательный аргумент)
{Реализация пирамидальной сортировки массива} % Подпись листинга

\clearpage

\section{Функциональные тесты}

В таблице \ref{tbl:func_tests} приведены функциональные тесты для разработанных алгоритмов сортировок. Все тесты пройдены успешно.

\begin{table}[ht]
	\small
	\begin{center}
		\begin{threeparttable}
			\caption{Функциональные тесты}
			\label{tbl:func_tests}
			\begin{tabular}{|c|c|c|}
				\hline
				\bfseries Массив
				& \bfseries Ожидаемый результат
				& \bfseries Фактический результат \\ 
				\hline
				[1, 2, 3, 4, 5] & [1, 2, 3, 4, 5] & [1, 2, 3, 4, 5] \\
				\hline
				[5, 4, 3, 2, 1]  & [1, 2, 3, 4, 5] & [1, 2, 3, 4, 5] \\
				\hline
				[~]  & [~] & [~] \\
				\hline
				[1]  & [1] & [1]\\
				\hline
				[4, 1, 2, 3]  & [1, 2, 3, 4] & [1, 2, 3, 4] \\
				\hline
				[2, 1]  & [1, 2] & [1, 2] \\
				\hline
				[31, 57, 24, -10, 59]  & [-10, 24, 31, 57, 59] & [-10, 24, 31, 57, 59] \\
				\hline
			\end{tabular}	
		\end{threeparttable}	
	\end{center}
\end{table}


\section*{Вывод}
Были разработаны и протестированы спроектированные алгоритмы сортировок:
Шелла, гномья и пирамидальная.