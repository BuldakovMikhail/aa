\chapter*{ВВЕДЕНИЕ}
\addcontentsline{toc}{chapter}{ВВЕДЕНИЕ}

Сортировка данных является фундаментальной задачей в области информатики и алгоритмов. 
Независимо от конкретной области применения, эффективные алгоритмы сортировки существенно влияют на производительность программных систем. 
От правильного выбора алгоритма зависит как время выполнения программы, так и затраты ресурсов компьютера \cite{knut}.

Алгоритмы сортировки находят применение в следующих сферах:
\begin{itemize}
	\item базы данных;
	\item анализ данных и статистика;
	\item алгоритмы машинного обучения;
	\item криптография.
\end{itemize}

Цель данной лабораторной работы --- рассмотреть алгоритмы сортировки.
Для достижения поставленной цели необходимо выполнить следующие задачи:
\begin{itemize}
	\item описать алгоритмы гномьей, пирамидальной сортировок и сортировку по алгоритму Шелла;
	\item разработать программное обеспечение, реализующее алгоритмы сортировок;
	\item выбрать инструменты для реализации и замера процессорного времени
	выполнения реализаций алгоритмов;
	\item проанализировать затраты реализаций алгоритмов по времени.
\end{itemize}
