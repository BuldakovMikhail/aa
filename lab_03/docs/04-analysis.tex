\chapter{Аналитический раздел}

Сортировкой называют перестановку объектов, при которой они располагаются в порядке возрастания или убывания \cite{knut}.

В данном разделе будут описаны три алгоритма сортировок: гномья, пирамидальная и Шелла.

\section{Алгоритм гномьей сортировки}
Данный алгоритм можно разделить на следующие шаги \cite{gnome}:

\begin{enumerate}
	\item сравнить текущий и предыдущий элементы;
	\item если они в правильном порядке, сделать шаг на один горшок вперед, иначе поменять их местами и сделать шаг на один элемент назад;
	\item если нет предыдущего элемента, сделать шаг вперед;
	\item если нет следующего элемента, то закончить.
\end{enumerate}


\section{Алгоритм пирамидальной сортировки}

В основе данного алгоритма лежит принцип работы структуры данных куча \cite{heap}.
Данный алгоритм можно разделить на следующие шаги.

\begin{enumerate}
	\item Создать кучу на основе входного массива.
	\item Повторять следующие шаги до тех пор, пока куча не будет содержать только один элемент:
	\begin{itemize}
		\item поменять местами корневой элемент кучи (который является самым большим элементом) с последним элементом кучи;
		\item удалить последний элемент кучи (который теперь находится в правильном положении);
		\item сгруппировать оставшиеся элементы в кучу.
	\end{itemize}
	\item Отсортированный массив получается путем изменения порядка элементов во входном массиве.
\end{enumerate}


\section{Алгоритм Шелла}


Алгоритм Шелла может рассматриваться и как обобщение пузырьковой сортировки, так и сортировки вставками \cite{shell}.

Данный алгоритм можно разделить на следующие шаги.

\begin{enumerate}
	\item Выбрать некоторый интервал (шаг). Обычно начальный шаг выбирают равным половине длины массива.
	\item Сортировка вставками элементов, расположенных на расстоянии заданного шага друг от друга.
	\item Уменьшение шага вдвое и повтор шага 2. Процесс повторяется до тех пор, пока шаг не станет равным 1.
	\item Сортировка завершается с использованием обычной сортировки вставками (шаг равен 1).
\end{enumerate}


\section*{Вывод}

В данном разделе были описаны три алгоритма сортировок: гномья, пирамидальная и Шелла.
