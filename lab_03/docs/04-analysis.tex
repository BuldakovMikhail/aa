\chapter{Аналитический раздел}

Сортировкой называют перестановку объектов, при которой они располагаются в порядке возрастания или убывания \cite{knut}.

В данном разделе будут описаны три алгоритма сортировок: гномья, пирамидальная и Шелла.

\section{Алгоритм гномьей сортировки}
Данный алгоритм можно разделить на следующие шаги \cite{gnome}:

\begin{itemize}
	\item сравнить текущий и предыдущий элементы;
	\item если они в правильном порядке, сделать шаг на один элемент вперед, иначе поменять их местами и сделать шаг на один элемент назад;
	\item если нет предыдущего элемента, сделать шаг вперед;
	\item если нет следующего элемента, то закончить.
\end{itemize}


\section{Алгоритм пирамидальной сортировки}

В основе данного алгоритма лежит принцип работы структуры данных куча \cite{heap}.
Данный алгоритм можно разделить на следующие шаги:

\begin{itemize}
	\item создать кучу на основе входного массива;
	\item повторять следующие шаги до тех пор, пока куча не будет содержать только один элемент:
	\begin{enumerate}
		\item поменять местами корневой элемент кучи (который является самым большим элементом) с последним элементом кучи;
		\item удалить последний элемент кучи (который теперь находится в правильном положении);
		\item сгруппировать оставшиеся элементы в кучу;
	\end{enumerate}
	\item отсортированный массив получается путем изменения порядка элементов во входном массиве.
\end{itemize}


\section{Алгоритм Шелла}


Данный алгоритм можно разделить на следующие шаги \cite{shell}:

\begin{itemize}
	\item выбрать некоторый интервал (шаг). Обычно начальный шаг выбирают равным половине длины массива;
	\item сортировка вставками элементов, расположенных на расстоянии заданного шага друг от друга;
	\item уменьшение шага вдвое и повтор шага 2. Процесс повторяется до тех пор, пока шаг не станет равным 1;
	\item сортировка завершается с использованием обычной сортировки вставками (шаг равен 1).
\end{itemize}


\section*{Вывод}

В данном разделе были описаны три алгоритма сортировок: гномья, пирамидальная и Шелла.
