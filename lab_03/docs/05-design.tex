\chapter{Конструкторский раздел}

В этом разделе будет представлено описание используемых типов данных, а также схематические изображения алгоритмов сортировок: гномьей, пирамидальной и Шелла.

\section{Требования к программному обеспечению}

Программа должна поддерживать два режима работы: режим массового замера времени и режим сортировки введенного массива.

Режим массового замера времени должен обладать следующей функциональностью:
\begin{itemize}
	\item генерировать массивы различного размер для проведения замеров;
	\item осуществлять массовый замер, используя сгенерированные данные;
	\item результаты массового замера должны быть представлены в виде таблицы и графика.
\end{itemize}

К режиму сортировки выдвигается ряд требований:
\begin{itemize}
	\item возможность работать с массивами разного размера, которые вводит пользователь;
	\item наличие интерфейса для выбора действий;
	\item на выходе программы, массив отсортированный тремя алгоритмами по возрастанию.
\end{itemize}

\section{Описание используемых типов данных}

При реализации алгоритмов будут использованы следующие структуры и типы данных:
\begin{itemize}
	\item целое число представляет количество элементов в массиве;
	\item массив вещественных чисел;
	\item куча представляется с помощью массива вещественных чисел.
\end{itemize}

\section{Разработка алгоритмов}

На рисунке \ref{img:gnome} представлена схема алгоритма гномьей сортировки. 
На рисунке \ref{img:shell} представлена схема алгоритма гномьей сортировки. 
На рисунке \ref{img:heapsort} представлена схема алгоритма пирамидальной сортировки. 
На рисунке \ref{img:heapify} представлена схема алгоритма вспомогательной подпрограммы, строящей кучу.
\clearpage

\includeimage
{gnome} % Имя файла без расширения (файл должен быть расположен в директории inc/img/)
{f} % Обтекание (без обтекания)
{h} % Положение рисунка (см. figure из пакета float)
{1\textwidth} % Ширина рисунка
{Схема алгоритма гномьей сортировки} % Подпись рисунка

\includeimage
{shell} % Имя файла без расширения (файл должен быть расположен в директории inc/img/)
{f} % Обтекание (без обтекания)
{h} % Положение рисунка (см. figure из пакета float)
{0.8\textwidth} % Ширина рисунка
{Схема алгоритма сортировки Шелла} % Подпись рисунка


\includeimage
{heapsort} % Имя файла без расширения (файл должен быть расположен в директории inc/img/)
{f} % Обтекание (без обтекания)
{h} % Положение рисунка (см. figure из пакета float)
{1\textwidth} % Ширина рисунка
{Схема алгоритма пирамидальной сортировки} % Подпись рисунка

\includeimage
{heapify} % Имя файла без расширения (файл должен быть расположен в директории inc/img/)
{f} % Обтекание (без обтекания)
{h} % Положение рисунка (см. figure из пакета float)
{1\textwidth} % Ширина рисунка
{Схема алгоритма пирамидальной сортировки} % Подпись рисунка
