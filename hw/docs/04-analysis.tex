\chapter{Аналитический раздел}

В данном разделе будет представлена информация о графовых моделях и приведены графовые модели поставленной задачи.


\section{Графовые модели программы}

Программа представлена в виде графа: набор вершин и множество соединяющих их направленных дуг.


Выделяют 2 типа дуг:
\begin{enumerate}
	\item операционное отношение~--- по передаче управления;
	\item информационное отношение~--- по передаче данных.
\end{enumerate}

Граф управления~--- модель, в который \textbf{вершины}~---операторы, \textbf{дуги}~--- операционные отношения.

Информационный граф~--- модель, в которой \textbf{вершины}: операторы, \textbf{дуги}~--- информационные отношения.

Операционная история~--- модель, в которой \textbf{вершины}: срабатывание операторов, \textbf{дуги}~--- операционные отношения.

Информационная история~--- модель, в которой \textbf{вершины}: срабатывание операторов, \textbf{дуги}~--- информационные отношения.

Графы более компактны, однако менее информативны, чем истории. Истории менее комактны, однако более информативны, чем графы.

На листинге \ref{lst:main.py} приведена реализации функции, вычисляющей расстояние Левенштейна.

\clearpage

\includelistingpretty
{main.py} % Имя файла с расширением (файл должен быть расположен в директории inc/lst/)
{python} % Язык программирования (необязательный аргумент)
{Реализация алгоритма вычисления расстояния Левенштейна} % Подпись листинга

\clearpage


На рисунке \ref{img:gu} представлен граф управления.
На рисунке \ref{img:ig} представлен информационный граф.

\includeimage
{gu} % Имя файла без расширения (файл должен быть расположен в директории inc/img/)
{f} % Обтекание (без обтекания)
{H} % Положение рисунка (см. figure из пакета float)
{1\textwidth} % Ширина рисунка
{Граф управления} % Подпись рисунка


\includeimage
{ig} % Имя файла без расширения (файл должен быть расположен в директории inc/img/)
{f} % Обтекание (без обтекания)
{H} % Положение рисунка (см. figure из пакета float)
{1\textwidth} % Ширина рисунка
{Информационный граф} % Подпись рисунка


Приведем графы для строк <<аба>> и <<ааа>>. На рисунке \ref{img:oi} представлена операционная история. На рисунке \ref{img:ii} представлена информационная история.

\includeimage
{oi} % Имя файла без расширения (файл должен быть расположен в директории inc/img/)
{f} % Обтекание (без обтекания)
{H} % Положение рисунка (см. figure из пакета float)
{1\textwidth} % Ширина рисунка
{Операционная история} % Подпись рисунка


\includeimage
{ii} % Имя файла без расширения (файл должен быть расположен в директории inc/img/)
{f} % Обтекание (без обтекания)
{H} % Положение рисунка (см. figure из пакета float)
{1\textwidth} % Ширина рисунка
{Информационная история} % Подпись рисунка

\section*{Вывод}

В реализуемом алгоритме наблюдается скошенный параллелизм. Можно вычислять диагонали, параллельные побочной в матрице Левенштейна, в различных потоках, поскольку элементы диагонали не зависят друг от друга.
