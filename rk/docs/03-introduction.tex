\chapter*{ВВЕДЕНИЕ}
\addcontentsline{toc}{chapter}{ВВЕДЕНИЕ}

С развитием вычислительных систем появилась потребность в параллельной обработке данных для повышения эффективности систем, ускорения вычислений и более рационального использования имеющихся ресурсов. 
Благодаря совершенствованию процессоров стало возможно использовать их для выполнения множества одновременных задач, что привело к появлению понятия «многопоточность» \cite{intro}.

Задание рубежного контроля --- многопоточное исправление орфографических ошибок.

Цель данного рубежного контроля --- описать принципы параллельных вычислений на основе нативных потоков для исправления орфографических ошибок в тексте. Для достижения поставленной цели необходимо выполнить следующие задачи:
\begin{itemize}
	\item описать алгоритм исправления орфографических ошибок в тексте;
	\item спроектировать программное обеспечение, реализующее алгоритм и его параллельную версию;
	\item выбрать инструменты для реализации и замера процессорного времени
	выполнения реализаций алгоритмов.
\end{itemize}
