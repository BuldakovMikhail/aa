\chapter{Конструкторский раздел}

В этом разделе будет представлено описание используемых типов данных, а также схематические изображения алгоритмов матричного умножения - стандартного, Штрассена, алгоритма Винограда и оптимизированного алгоритма Винограда.

\section{Требования к программному обеспечению}

Программа должна поддерживать два режима работы: режим массового замера времени и режим умножения матриц.

Режим массового замера времени должен обладать следующим функционалом:
\begin{itemize}
	\item генерировать матрицы различного размер для проведения замеров;
	\item осуществлять массовый замер, используя сгенерированные данные;
	\item результаты массового замера должны быть представлены в виде таблицы и графика.
\end{itemize}

К режиму умножения матриц выдвигается ряд требований:
\begin{itemize}
	\item возможность работать с матрицами разного размера, которые вводит пользователь;
	\item наличие интерфейса для выбора действий;
	\item проверять возможность умножения матриц.
\end{itemize}

\section{Разработка алгоритмов}

На рисунке \ref{img:classic} представлена схема классического алгоритма, выполняющего умножение двух матриц. На рисунках \ref{img:Vinograd} -- \ref{img:colfactor} изображены схемы алгоритма Винограда без оптимизаций. На рисунках \ref{img:strassen} и \ref{img:msr} изображены схемы алгоритмов умножения матриц Штрассена.  На рисунках \ref{img:VinogradOpt} -- \ref{img:colfactorOpt} изображены схемы алгоритма Винограда с оптимизациями.

\includeimage
{classic} % Имя файла без расширения (файл должен быть расположен в директории inc/img/)
{f} % Обтекание (без обтекания)
{h} % Положение рисунка (см. figure из пакета float)
{1\textwidth} % Ширина рисунка
{Классический алгоритм умножения матриц} % Подпись рисунка


\includeimage
{Vinograd} % Имя файла без расширения (файл должен быть расположен в директории inc/img/)
{f} % Обтекание (без обтекания)
{h} % Положение рисунка (см. figure из пакета float)
{0.9\textwidth} % Ширина рисунка
{Алгоритм умножения матриц Винограда} % Подпись рисунка

\includeimage
{rowfactor} % Имя файла без расширения (файл должен быть расположен в директории inc/img/)
{f} % Обтекание (без обтекания)
{h} % Положение рисунка (см. figure из пакета float)
{1\textwidth} % Ширина рисунка
{Вспомогательная подпрограмма, вычисляющая массив вспомогательных значений по строкам} % Подпись рисунка

\includeimage
{colfactor} % Имя файла без расширения (файл должен быть расположен в директории inc/img/)
{f} % Обтекание (без обтекания)
{h} % Положение рисунка (см. figure из пакета float)
{1\textwidth} % Ширина рисунка
{Вспомогательная подпрограмма, вычисляющая массив вспомогательных значений по столбцам} % Подпись рисунка

\includeimage
{strassen} % Имя файла без расширения (файл должен быть расположен в директории inc/img/)
{f} % Обтекание (без обтекания)
{h} % Положение рисунка (см. figure из пакета float)
{1\textwidth} % Ширина рисунка
{Алгоритм умножения матриц Штрассена} % Подпись рисунка

\includeimage
{msr} % Имя файла без расширения (файл должен быть расположен в директории inc/img/)
{f} % Обтекание (без обтекания)
{h} % Положение рисунка (см. figure из пакета float)
{1\textwidth} % Ширина рисунка
{Подпрограмма MSR, вычисляющая результат умножения матриц по алгоритму Штрассена} % Подпись рисунка

\includeimage
{VinogradOpt} % Имя файла без расширения (файл должен быть расположен в директории inc/img/)
{f} % Обтекание (без обтекания)
{h} % Положение рисунка (см. figure из пакета float)
{0.9\textwidth} % Ширина рисунка
{Алгоритм умножения матриц Винограда} % Подпись рисунка

\includeimage
{rowfactorOpt} % Имя файла без расширения (файл должен быть расположен в директории inc/img/)
{f} % Обтекание (без обтекания)
{h} % Положение рисунка (см. figure из пакета float)
{1\textwidth} % Ширина рисунка
{Вспомогательная подпрограмма, вычисляющая массив вспомогательных значений по строкам} % Подпись рисунка

\includeimage
{colfactorOpt} % Имя файла без расширения (файл должен быть расположен в директории inc/img/)
{f} % Обтекание (без обтекания)
{h} % Положение рисунка (см. figure из пакета float)
{1\textwidth} % Ширина рисунка
{Вспомогательная подпрограмма, вычисляющая массив вспомогательных значений по столбцам} % Подпись рисунка

\clearpage

\section{Оценка трудоемкости алгоритмов}

Введем модель для оценки трудоемкости алгоритмов:
\begin{enumerate}
	\item $+, -, =, +=, -=, ==, ||, \&\&, <, >, <=, >=, <<, >>, []$ --- считаем, что эти операции обладают трудоемкостью в 1 единицу;
	\item $*, /, *=, /=, \% $ --- считаем, что эти операции обладают трудоемкостью в 2 единицы;
	\item трудоемкость условного перехода примем за $0$.
\end{enumerate}

\subsection{Трудоемкость классического алгоритма}


