\chapter{Конструкторский раздел}

В этом разделе будет представлено описание используемых типов данных, а также схематические изображения алгоритмов матричного умножения - стандартного, Штрассена, алгоритма Винограда и оптимизированного алгоритма Винограда.

\section{Требования к программному обеспечению}

Программа должна поддерживать два режима работы: режим массового замера времени и режим умножения матриц.

Режим массового замера времени должен обладать следующим функционалом:
\begin{itemize}
	\item генерировать матрицы различного размер для проведения замеров;
	\item осуществлять массовый замер, используя сгенерированные данные;
	\item результаты массового замера должны быть представлены в виде таблицы и графика.
\end{itemize}

К режиму умножения матриц выдвигается ряд требований:
\begin{itemize}
	\item возможность работать с матрицами разного размера, которые вводит пользователь;
	\item наличие интерфейса для выбора действий;
	\item проверять возможность умножения матриц.
\end{itemize}

\includeimage
{classic} % Имя файла без расширения (файл должен быть расположен в директории inc/img/)
{f} % Обтекание (без обтекания)
{h} % Положение рисунка (см. figure из пакета float)
{1\textwidth} % Ширина рисунка
{Классический алгоритм умножения матриц} % Подпись рисунка
