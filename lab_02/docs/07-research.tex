\chapter{Исследовательский раздел}

В данном разделе будут приведены примеры работы программ, постановка эксперимента и сравнительный анализ алгоритмов на основе полученных данных.

\section{Демонстрация работы программы}


На рисунке \ref{img:program} представлена демонстрация работы разработанного программного обеспечения, а именно показаны результаты умножения матриц $A = \begin{pmatrix}
	1 & 2 & 3\\
	4 & 5 & 6 \\
	7 & 8 & 9\\
\end{pmatrix}$ и $B = \begin{pmatrix}
	3 & 6 & 7\\
	2 & 5 & 8 \\
	1 & 4 & 9 \\
\end{pmatrix}$.  
\clearpage

\includeimage
{program} % Имя файла без расширения (файл должен быть расположен в директории inc/img/)
{f} % Обтекание (без обтекания)
{h} % Положение рисунка (см. figure из пакета float)
{1\textwidth} % Ширина рисунка
{Демонстрация работы программы при умножении двух матриц} % Подпись рисунка

\clearpage


\section{Технические характеристики}

Технические характеристики устройства, на котором выполнялись замеры по времени.

\begin{itemize}
	\item Процессор: AMD Ryzen 5 4600H 3 ГГц \cite{amd}.
	\item Оперативная память: 16 ГБайт.
	\item Операционная система: Windows 10 Pro 64-разрядная система версии 22H2 \cite{windows}.
\end{itemize}

При замерах времени ноутбук был включен в сеть электропитания и был нагружен только системными приложениями.

\section{Время выполнения реализаций алгоритмов}

Результаты замеров времени выполнения реализаций алгоритмов умножения матриц приведены в таблицах \ref{tbl:time_measurements} -- \ref{tbl:time_measurements_odd}.
Замеры времени проводились на квадратных матрицах одного порядка и усреднялись для каждого набора одинаковых экспериментов.
В таблицах \ref{tbl:time_measurements} -- \ref{tbl:time_measurements_odd} используются следующие обозначения: 
\begin{itemize}
	\item К --- реализация классического алгоритма умножения матриц;
	\item В --- реализация алгоритма Винограда умножения матриц;
	\item ВО --- реализация алгоритма Винограда с оптимизацией умножения матриц;
	\item Ш --- реализация алгоритма Штрассена умножения матриц.
\end{itemize}

\begin{table}[h]
	\begin{center}
		\begin{threeparttable}
			\captionsetup{justification=raggedright,singlelinecheck=off}
			\caption{Время работы реализации алгоритмов (в с)}
			\label{tbl:time_measurements}
			\begin{tabular}{|c|c|c|c|c|}
				\hline
				Порядок матриц &  К  & В & ВО & Ш \\
				\hline
			6 &$ 1.875* 10^{-4} $&$ 2.031* 10^{-4} $&$ 1.719* 10^{-4} $&$ 1.547* 10^{-3}$\\
			\hline
			11 &$ 1.047* 10^{-3} $&$ 1.187* 10^{-3} $&$ 8.750* 10^{-4} $&$ 1.105* 10^{-2}$\\
			\hline
			16 &$ 3.203* 10^{-3} $&$ 3.359* 10^{-3} $&$ 2.578* 10^{-3} $&$ 1.095* 10^{-2}$\\
			\hline
			21 &$ 7.328* 10^{-3} $&$ 7.938* 10^{-3} $&$ 5.656* 10^{-3} $&$ 7.731* 10^{-2}$\\
			\hline
			26 &$ 1.366* 10^{-2} $&$ 1.406* 10^{-2} $&$ 1.034* 10^{-2} $&$ 7.767* 10^{-2}$\\
			\hline
			31 &$ 2.289* 10^{-2} $&$ 2.447* 10^{-2} $&$ 1.772* 10^{-2} $&$ 7.734* 10^{-2}$\\
			\hline
			36 &$ 3.605* 10^{-2} $&$ 3.692* 10^{-2} $&$ 2.680* 10^{-2} $&$ 5.358* 10^{-1}$\\
			\hline
			41 &$ 5.269* 10^{-2} $&$ 5.394* 10^{-2} $&$ 4.081* 10^{-2} $&$ 5.314* 10^{-1}$\\
			\hline
			46 &$ 7.452* 10^{-2} $&$ 7.644* 10^{-2} $&$ 5.633* 10^{-2} $&$ 5.323* 10^{-1}$\\
			\hline
			51 &$ 1.003* 10^{-1} $&$ 1.023* 10^{-1} $&$ 7.473* 10^{-2} $&$ 5.264* 10^{-1}$\\
			\hline
			\end{tabular}
		\end{threeparttable}
	\end{center}
\end{table}

\begin{table}[h]
	\begin{center}
		\begin{threeparttable}
			\captionsetup{justification=raggedright,singlelinecheck=off}
			\caption{Время работы реализации алгоритмов для четных порядков матриц (в с)}
			\label{tbl:time_measurements_even}
			\begin{tabular}{|c|c|c|c|c|}
				\hline
				Порядок матриц &  К  & В & ВО & Ш \\
				\hline
				6 &$ 1.563* 10^{-4} $&$ 2.188* 10^{-4} $&$ 1.563* 10^{-4} $&$ 1.531* 10^{-3}$\\
				\hline
				10 &$ 8.125* 10^{-4} $&$ 8.750* 10^{-4} $&$ 6.875* 10^{-4} $&$ 1.066* 10^{-2}$\\
				\hline
				14 &$ 2.094* 10^{-3} $&$ 2.250* 10^{-3} $&$ 1.750* 10^{-3} $&$ 1.097* 10^{-2}$\\
				\hline
				18 &$ 4.437* 10^{-3} $&$ 4.687* 10^{-3} $&$ 3.531* 10^{-3} $&$ 7.587* 10^{-2}$\\
				\hline
				22 &$ 8.063* 10^{-3} $&$ 8.406* 10^{-3} $&$ 6.281* 10^{-3} $&$ 7.444* 10^{-2}$\\
				\hline
				26 &$ 1.325* 10^{-2} $&$ 1.394* 10^{-2} $&$ 1.050* 10^{-2} $&$ 7.503* 10^{-2}$\\
				\hline
				30 &$ 2.041* 10^{-2} $&$ 2.125* 10^{-2} $&$ 1.588* 10^{-2} $&$ 7.553* 10^{-2}$\\
				\hline
				34 &$ 2.981* 10^{-2} $&$ 3.041* 10^{-2} $&$ 2.244* 10^{-2} $&$ 5.243* 10^{-1}$\\
				\hline
				38 &$ 4.141* 10^{-2} $&$ 4.216* 10^{-2} $&$ 3.103* 10^{-2} $&$ 5.255* 10^{-1}$\\
				\hline
				42 &$ 5.688* 10^{-2} $&$ 5.719* 10^{-2} $&$ 4.206* 10^{-2} $&$ 5.244* 10^{-1}$\\
				\hline
				46 &$ 7.341* 10^{-2} $&$ 7.444* 10^{-2} $&$ 5.506* 10^{-2} $&$ 5.239* 10^{-1}$\\
				\hline
				50 &$ 9.425* 10^{-2} $&$ 9.547* 10^{-2} $&$ 7.016* 10^{-2} $&$ 5.277* 10^{-1}$\\
				\hline
			\end{tabular}
		\end{threeparttable}
	\end{center}
\end{table}

\begin{table}[h]
	\begin{center}
		\begin{threeparttable}
			\captionsetup{justification=raggedright,singlelinecheck=off}
			\caption{Время работы реализации алгоритмов для нечетных порядков матриц (в с)}
			\label{tbl:time_measurements_odd}
			\begin{tabular}{|c|c|c|c|c|}
				\hline
				Порядок матриц &  К  & В & ВО & Ш \\
				\hline
			7 &$ 2.812* 10^{-4} $&$ 3.437* 10^{-4} $&$ 2.812* 10^{-4} $&$ 1.625* 10^{-3}$\\
			\hline
			11 &$ 1.031* 10^{-3} $&$ 1.219* 10^{-3} $&$ 8.750* 10^{-4} $&$ 1.078* 10^{-2}$\\
			\hline
			15 &$ 2.563* 10^{-3} $&$ 2.781* 10^{-3} $&$ 2.125* 10^{-3} $&$ 1.066* 10^{-2}$\\
			\hline
			19 &$ 5.188* 10^{-3} $&$ 5.500* 10^{-3} $&$ 4.156* 10^{-3} $&$ 7.531* 10^{-2}$\\
			\hline
			23 &$ 9.469* 10^{-3} $&$ 9.875* 10^{-3} $&$ 7.250* 10^{-3} $&$ 7.550* 10^{-2}$\\
			\hline
			27 &$ 1.512* 10^{-2} $&$ 1.584* 10^{-2} $&$ 1.169* 10^{-2} $&$ 7.484* 10^{-2}$\\
			\hline
			31 &$ 2.266* 10^{-2} $&$ 2.344* 10^{-2} $&$ 1.756* 10^{-2} $&$ 7.541* 10^{-2}$\\
			\hline
			35 &$ 3.275* 10^{-2} $&$ 3.334* 10^{-2} $&$ 2.488* 10^{-2} $&$ 5.311* 10^{-1}$\\
			\hline
			39 &$ 4.462* 10^{-2} $&$ 4.612* 10^{-2} $&$ 3.466* 10^{-2} $&$ 5.259* 10^{-1}$\\
			\hline
			43 &$ 6.222* 10^{-2} $&$ 6.256* 10^{-2} $&$ 4.534* 10^{-2} $&$ 5.279* 10^{-1}$\\
			\hline
			47 &$ 7.841* 10^{-2} $&$ 7.928* 10^{-2} $&$ 6.081* 10^{-2} $&$ 5.288* 10^{-1}$\\
			\hline
			51 &$ 1.037* 10^{-1} $&$ 1.059* 10^{-1} $&$ 7.500* 10^{-2} $&$ 5.279* 10^{-1}$\\
			\hline
			\end{tabular}
		\end{threeparttable}
	\end{center}
\end{table}

На рисунках \ref{img:figure} -- \ref{img:figureOdd} изображены графики сравнения реализаций алгоритмов по времени.

\includeimage
{figure} % Имя файла без расширения (файл должен быть расположен в директории inc/img/)
{f} % Обтекание (без обтекания)
{h} % Положение рисунка (см. figure из пакета float)
{1\textwidth} % Ширина рисунка
{Сравнение алгоритмов по времени} % Подпись рисунка

\includeimage
{figureEven} % Имя файла без расширения (файл должен быть расположен в директории inc/img/)
{f} % Обтекание (без обтекания)
{h} % Положение рисунка (см. figure из пакета float)
{1\textwidth} % Ширина рисунка
{Сравнение алгоритмов по времени для четных порядков матриц} % Подпись рисунка

\includeimage
{figureOdd} % Имя файла без расширения (файл должен быть расположен в директории inc/img/)
{f} % Обтекание (без обтекания)
{h} % Положение рисунка (см. figure из пакета float)
{1\textwidth} % Ширина рисунка
{Сравнение алгоритмов по времени для нечетных порядков матриц} % Подпись рисунка


\clearpage

\section*{Вывод}

В результате замеров времени выполнения реализаций различных алгоритмов было получено, что для матриц с порядком 51, реализация алгоритма Винограда оказалась в 1.02 раза хуже реализации классического алгоритма по времени выполнения, при этом реализация алгоритма Винограда с оптимизациями оказалась лучше в 1.34 раза реализации классического алгоритма и в 1.37 раз лучше реализации без оптимизаций по времени выполнения. 
Для матриц с четным порядком 46, реализация алгоритма Винограда хуже реализации классического алгоритма в 1.02 раза по времени выполнения, а реализация алгоритма Винограда с оптимизацией оказалась лучше классического алгоритма в 1.32 раза. 

Для матриц с порядком 51, реализация алгоритма Штрассена оказалась хуже по времени выполнения в 5.24 раза, в 5.14 раза и в 7.04 раза, чем реализации классического алгоритма, алгоритма Винограда и алгоритма Винограда с оптимизациями соответственно. 
Такой результат обусловлен тем, что в данной реализации много времени тратится на выделение подматриц, их сложение и вычитание, а также на рекурсивные вызовы. По теоретической оценке трудоемкости, можно заметить насколько велика константа перед $n^{\log_{2}7}$.

Стоит заметить, что на графике замеров времени выполнения реализаций алгоритмов \ref{img:figure} у реализации алгоритма Штрассена присутствуют отчетливые ступеньки. Это обусловлено особенностью алгоритма, который может работать только с квадратными матрицами порядка $2^k$, а матрицы других размеров приводятся к данному размеру дополнением нулевыми строками и столбцами. Таким образом, разные размеры $n$ могу приводится к одному новому порядку $2^k$, что в свою очередь приводит к почти идентичному времени обработки.

Полученные на практике результаты примерно соответствуют теоретическим оценкам.