\chapter{Аналитический раздел}

Матрицей называется прямоугольная таблица чисел, вида \eqref{eq:matrix}, состоящая из $m$ строк и $n$ столбцов \cite{matrix}.

\begin{equation}
	\label{eq:matrix}
	\begin{pmatrix}
		a_{11} & a_{12} & \ldots & a_{1n}\\
		a_{21} & a_{22} & \ldots & a_{2n}\\
		\vdots & \vdots & \ddots & \vdots\\
		a_{m1} & a_{m2} & \ldots & a_{mn}
	\end{pmatrix},
\end{equation}

Пусть $A$ -- матрица, тогда $а_{ij}$ -- элемент этой матрицы, который находится на \textit{i-ой} строке и \textit{j-ом} столбце.

Если количество столбцов первой матрицы совпадает с количеством строк второй матрицы, то возможно выполнить их матричное умножение. В результате умножения получится матрица-произведение, количество строк в которой равно количеству строк первой матрицы, а количество столбцов равно количеству столбцов второй матрицы.

\section{Классический алгоритм}

Пусть даны две прямоугольные матрицы $A$ и $B$ размеров $[m \times n]$ и $[n \times k]$ соответственно. В результате произведение матриц $A$ и $B$ получим матрицу $C$ размера $[m \times k]$, элементы которой вычисляются по \eqref{eq:matrix_classic}.

\begin{equation}
	\label{eq:matrix_classic}
	c_{ij} = \sum_{l=1}^{n}a_{il}b_{lj}
\end{equation}

Классический алгоритм умножения матриц, реализует формулу \eqref{eq:matrix_classic}.