\chapter*{ЗАКЛЮЧЕНИЕ}
\addcontentsline{toc}{chapter}{ЗАКЛЮЧЕНИЕ}

Цель данной лабораторной работы была достигнута, а именно были исследованы алгоритмы умножения матриц.


Для достижения поставленной цели были выполнены следующие задачи.
\begin{enumerate}
	\item Описаны следующие алгоритмы умножения матриц:
	\begin{itemize}
		\item классический алгоритм;
		\item алгоритм Винограда;
		\item алгоритм Штрассена.
	\end{itemize}
	\item Проведена оптимизация алгоритма Винограда.
	\item Разработано программное обеспечение, реализующее алгоритмы умножения.
	\item Выбраны инструменты для реализации алгоритмов и замера процессорного времени их выполнения.
	\item Проведен анализ затрат реализаций алгоритмов по времени. 
\end{enumerate}

В результате исследования реализаций алгоритмов было выявлено, что для матриц с порядком 51, реализация алгоритма Винограда оказалась в 1.02 раза хуже реализации классического алгоритма по времени выполнения, при этом реализация алгоритма Винограда с оптимизациями оказалась лучше в 1.34 раза реализации классического алгоритма и в 1.37 раз лучше реализации без оптимизаций по времени выполнения. 
Для матриц с четным порядком 46, реализация алгоритма Винограда хуже реализации классического алгоритма в 1.02 раза по времени выполнения, а реализация алгоритма Винограда с оптимизацией оказалась лучше классического алгоритма в 1.32 раза. 

Для матриц с порядком 51, реализация алгоритма Штрассена оказалась хуже по времени выполнения в 5.24 раза, в 5.14 раза и в 7.04 раза, чем реализации классического алгоритма, алгоритма Винограда и алгоритма Винограда с оптимизациями соответственно. 