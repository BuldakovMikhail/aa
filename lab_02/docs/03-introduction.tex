\chapter*{ВВЕДЕНИЕ}
\addcontentsline{toc}{chapter}{ВВЕДЕНИЕ}

Матрицы являются одним из основных инструментов линейной алгебры, они позволяют описывать и анализировать линейные отношения между различными объектами и явлениями. В настоящее время матрицы широко используются в науке, технике, экономике и других сферах человеческой деятельности.

Размеры матриц могут быть очень большими в зависимости от конкретной задачи, поэтому оптимизация алгоритмов обработки матриц является важной задачей программирования. Основной акцент будет сделан на оптимизации алгоритма умножения матриц.

Цель данной лабораторной работы -- описать, реализовать и исследовать алгоритмы умножения матриц и их оптимизации.
Для достижения поставленной цели необходимо выполнить следующие задачи.
\begin{enumerate}
	\item Изучить алгоритмы умножения матриц: 
	\begin{itemize}
		\item классический алгоритм;
		\item алгоритм Винограда;
		\item алгоритм Штрассена.
	\end{itemize}
	\item Оптимизировать перечисленные алгоритмы.
	\item Разработать программное обеспечение, реализующее алгоритмы умножения.
	\item Выбрать инструменты для реализации и замера процессорного времени
	выполнения алгоритмов.
	\item Проанализировать затраты реализаций алгоритмов по времени и по памяти.
\end{enumerate}
