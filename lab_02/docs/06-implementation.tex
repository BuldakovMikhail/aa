\chapter{Технологический раздел}

В данном разделе будут приведены требования к программному обеспечению, средства реализации, листинг кода и функциональные тесты.


\section{Средства реализации}

Для реализации данной работы был выбран язык \textit{Python}~\cite{python}.
Такой выбор обусловлен опытом работы с этим языком программирования.
Также данный язык позволяет замерять процессорное время с помощью модуля \textit{time} и в нем присутствует библиотека \textbf{numpy} для удобной работы с матрицами. 

Процессорное время было замерено с помощью функции \textit{process\_time()} из модуля \textit{time}~\cite{python-time}.

\section{Сведения о модулях программы}

Данная программа разбита на следующие модули:
\begin{itemize}
	\item $main.py$ --- файл, содержащий функцию $main$;
	\item $algorithms.py$ --- файл, содержащий код реализаций всех алгоритмов умножения матриц;
	\item $compare\_time.py$ --- файл, в котором содержатся функции для замера и вывода времени выполнения реализаций алгоритмов.
\end{itemize}

\section{Реализация алгоритмов}

В листингах \ref{lst:classic.py} -- \ref{lst:vinogradOpt.py} приведены реализации классического алгоритма, алгоритмов Винограда с оптимизацией и без. Реализация алгоритма Штрассена приведена на листинге в приложении \ref{lst:strassen.py}.


\clearpage

\includelistingpretty
{classic.py} % Имя файла с расширением (файл должен быть расположен в директории inc/lst/)
{python} % Язык программирования (необязательный аргумент)
{Функция умножения матриц по классическому алгоритму} % Подпись листинга

\clearpage

\includelistingpretty
{vinograd.py} % Имя файла с расширением (файл должен быть расположен в директории inc/lst/)
{python} % Язык программирования (необязательный аргумент)
{Функция умножения матриц по алгоритму Винограда} % Подпись листинга

\clearpage

\includelistingpretty
{vinogradOpt.py} % Имя файла с расширением (файл должен быть расположен в директории inc/lst/)
{python} % Язык программирования (необязательный аргумент)
{Функция умножения матриц по алгоритму Винограда с оптимизацией} % Подпись листинга

\clearpage

\section{Функциональные тесты}

В таблице \ref{tbl:func_tests} приведены функциональные тесты для разработанных алгоритмов умножения матриц. Все тесты пройдены успешно.

\begin{table}[ht]
	\small
	\begin{center}
		\begin{threeparttable}
			\caption{Функциональные тесты для алгоритмов умножения матриц}
			\label{tbl:func_tests}
			\begin{tabular}{|c|c|c|c|c|}
				\hline
				\multicolumn{2}{|c|}{\bfseries Входные данные}
				& \multicolumn{2}{c|}{\bfseries Результат для всех алгоритмов} \\
				\hline 
				\bfseries Матрица 1
				& \bfseries Матрица 2
				& \bfseries Ожидаемый результат
				& \bfseries Фактический результат \\
				\hline
				$\begin{pmatrix}
					1 & 5 & 7\\
					2 & 6 & 8\\
					3 & 7 & 9
				\end{pmatrix}$ 
				&  
				$\begin{pmatrix}
					&
				\end{pmatrix}$
				&
				\text{Сообщение об ошибке}
				&
				\text{Сообщение об ошибке} \\ 
				\hline
				$\begin{pmatrix}
					1 & 5 & 7\\
				\end{pmatrix}$ 
				&  
				$\begin{pmatrix}
					1 & 2 & 3\\
				\end{pmatrix}$
				&
				\text{Сообщение об ошибке}
				&
				\text{Сообщение об ошибке} \\ 
				\hline
				$\begin{pmatrix}
					1 & 2 & 3\\
					4 & 5 & 6 \\
					7 & 8 & 9 \\
				\end{pmatrix}$ 
				&  
				$\begin{pmatrix}
					1 & 0 & 0\\
					0 & 1 & 0 \\
					0 & 0 & 1 \\
				\end{pmatrix}$
				&
				$\begin{pmatrix}
					1 & 2 & 3\\
					4 & 5 & 6 \\
					7 & 8 & 9 \\
				\end{pmatrix}$ 
				&
				$\begin{pmatrix}
					1 & 2 & 3\\
					4 & 5 & 6 \\
					7 & 8 & 9 \\
				\end{pmatrix}$ \\ 
				\hline
				$\begin{pmatrix}
					3 & 5\\
					2 & 1\\
					9 & 7\\
				\end{pmatrix}$
				&
				$\begin{pmatrix}
					1 & 2 & 3\\
					4 & 5 & 6 \\
				\end{pmatrix}$
				&
				$\begin{pmatrix}
					23 & 31 & 39\\
					6 & 9 & 12\\
					37 & 53 & 69 \\
				\end{pmatrix}$ 
				&
				$\begin{pmatrix}
					23 & 31 & 39\\
					6 & 9 & 12\\
					37 & 53 & 69 \\
				\end{pmatrix}$ \\
				\hline
				$\begin{pmatrix}
					10
				\end{pmatrix}$
				&
				$\begin{pmatrix}
					35
				\end{pmatrix}$
				&
				$\begin{pmatrix}
					350
				\end{pmatrix}$ 
				&
				$\begin{pmatrix}
					350
				\end{pmatrix}$ \\ 
				\hline
				$\begin{pmatrix}
					1 & 5 & 7\\
					2 & 6 & 8\\
					3 & 7 & 9
				\end{pmatrix}$ 
				&
				$\begin{pmatrix}
					1 & 2 & 3\\
					4 & 5 & 6 \\
					7 & 8 & 9 \\
				\end{pmatrix}$ 
				&
				$\begin{pmatrix}
					70 & 83 & 96\\
					82 & 98 & 114 \\
					94 & 113 & 132 \\
				\end{pmatrix}$ 
				&
				$\begin{pmatrix}
					70 & 83 & 96\\
					82 & 98 & 114 \\
					94 & 113 & 132 \\
				\end{pmatrix}$ \\ 
				\hline
			\end{tabular}
		\end{threeparttable}
	\end{center}
\end{table}


\section*{Вывод}
Были разработаны и протестированы спроектированные алгоритмы:
классического умножения матриц, алгоритм Винограда, оптимизированный алгоритм Винограда, а также алгоритм Штрассена.