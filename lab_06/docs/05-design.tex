\chapter{Конструкторский раздел}

В этом разделе будет представлено описание используемых типов данных, а также схематические изображения алгоритмов решения задачи коммивояжера.

\section{Требования к программному обеспечению}

Программа должна поддерживать два режима работы: режим массового замера времени и режим решения задачи коммивояжера.

Режим массового замера времени должен обладать следующей функциональностью:
\begin{itemize}
	\item генерировать графы различного размер для проведения замеров;
	\item осуществлять массовый замер, используя сгенерированные данные;
	\item результаты массового замера должны быть представлены в виде таблицы и графика.
\end{itemize}

К режиму решения задачи коммивояжера выдвигается следующий ряд требований:
\begin{itemize}
	\item возможность работать графами, записанными в файл;
	\item возможность вводить матрицы смежности графов;
	\item наличие интерфейса для выбора действий;
	\item на выходе программы, стоимость и маршрут кратчайшей длины.
\end{itemize}

\section{Описание используемых типов данных}

При реализации алгоритмов будут использованы следующие структуры и типы данных:
\begin{itemize}
	\item граф --- множество вершин и ребер между ними, задается с помощью матрицы смежности;
	\item матрица --- двумерный массив чисел.
\end{itemize}

\section{Разработка алгоритмов}

На рисунке \ref{img:brut} представлена схема алгоритма решения задачи коммивояжера полным перебором. На рисунке \ref{img:ant1} представлена схема муравьиного алгоритма. 

\clearpage

\includeimage
{brut} % Имя файла без расширения (файл должен быть расположен в директории inc/img/)
{f} % Обтекание (без обтекания)
{h} % Положение рисунка (см. figure из пакета float)
{0.75\textwidth} % Ширина рисунка
{Схема алгоритма решения задачи коммивояжера полным перебором} % Подпись рисунка

\clearpage

\includeimage
{ant1} % Имя файла без расширения (файл должен быть расположен в директории inc/img/)
{f} % Обтекание (без обтекания)
{h} % Положение рисунка (см. figure из пакета float)
{1\textwidth} % Ширина рисунка
{Схема муравьиного алгоритма} % Подпись рисунка


\clearpage

\includeimage
{getprobs} % Имя файла без расширения (файл должен быть расположен в директории inc/img/)
{f} % Обтекание (без обтекания)
{h} % Положение рисунка (см. figure из пакета float)
{1\textwidth} % Ширина рисунка
{Схема алгоритма вычисления массива вероятностей переходов в непосещенные города} % Подпись рисунка


\clearpage

\includeimage
{updatephero} % Имя файла без расширения (файл должен быть расположен в директории inc/img/)
{f} % Обтекание (без обтекания)
{h} % Положение рисунка (см. figure из пакета float)
{0.9\textwidth} % Ширина рисунка
{Схема алгоритма обновления матрицы феромона} % Подпись рисунка

