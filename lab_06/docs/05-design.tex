\chapter{Конструкторский раздел}

В этом разделе будет представлено описание используемых типов данных, а также схематические изображения алгоритмов решения задачи коммивояжера.

\section{Требования к программному обеспечению}

Программа должна поддерживать два режима работы: режим массового замера времени и режим решения задачи коммивояжера.

Режим массового замера времени должен обладать следующей функциональностью:
\begin{itemize}
	\item генерировать графы различного размер для проведения замеров;
	\item осуществлять массовый замер, используя сгенерированные данные;
	\item результаты массового замера должны быть представлены в виде таблицы и графика.
\end{itemize}

К режиму решения задачи коммивояжера выдвигается следующий ряд требований:
\begin{itemize}
	\item возможность вводить матрицы смежности графов;
	\item наличие интерфейса для выбора действий;
	\item на выходе программы, стоимость и маршрут кратчайшей длины.
\end{itemize}

\section{Описание используемых типов данных}

При реализации алгоритмов будут использованы следующие структуры и типы данных:
\begin{itemize}
	\item граф --- множество вершин и ребер между ними, задается с помощью матрицы смежности;
	\item матрица --- двумерный массив чисел.
\end{itemize}

\section{Разработка алгоритмов}

На рисунке \ref{img:brut} представлена схема алгоритма решения задачи коммивояжера полным перебором. На рисунке \ref{img:ant1} представлена схема муравьиного алгоритма. На рисунках \ref{img:getprobs} и \ref{img:updatephero} изображены схемы алгоритмов вспомогательных подпрограмм.

\clearpage

\includeimage
{brut} % Имя файла без расширения (файл должен быть расположен в директории inc/img/)
{f} % Обтекание (без обтекания)
{h} % Положение рисунка (см. figure из пакета float)
{0.75\textwidth} % Ширина рисунка
{Схема алгоритма решения задачи коммивояжера полным перебором} % Подпись рисунка

\clearpage

\includeimage
{ant1} % Имя файла без расширения (файл должен быть расположен в директории inc/img/)
{f} % Обтекание (без обтекания)
{h} % Положение рисунка (см. figure из пакета float)
{1\textwidth} % Ширина рисунка
{Схема муравьиного алгоритма} % Подпись рисунка


\clearpage

\includeimage
{getprobs} % Имя файла без расширения (файл должен быть расположен в директории inc/img/)
{f} % Обтекание (без обтекания)
{h} % Положение рисунка (см. figure из пакета float)
{1\textwidth} % Ширина рисунка
{Схема алгоритма вычисления массива вероятностей переходов в непосещенные города} % Подпись рисунка


\clearpage

\includeimage
{updatephero} % Имя файла без расширения (файл должен быть расположен в директории inc/img/)
{f} % Обтекание (без обтекания)
{h} % Положение рисунка (см. figure из пакета float)
{0.9\textwidth} % Ширина рисунка
{Схема алгоритма обновления матрицы феромона} % Подпись рисунка

\clearpage

\section{Оценка трудоемкости алгоритмов}

Модель для оценки трудоемкости алгоритмов состоит из шести пунктов:
\begin{enumerate}
	\item $+, -, =, +=, -=, ==, ||, \&\&, <, >, <=, >=, <<, >>, []$ --- считается, что эти операции обладают трудоемкостью в 1 единицу;
	\item $*, /, *=, /=, \%, ** $ --- считается, что эти операции обладают трудоемкостью в 2 единицы;
	\item трудоемкость условного перехода принимается за $0$;
	\item трудоемкость условного оператора рассчитывается по формуле \eqref{eq:if},
	\begin{equation}
		\label{eq:if}
		f_{if} = f_{\text{условия}} + 
		\begin{cases}
			min(f_1, f_2), & \text{лучший случай}\\
			max(f_1, f_2), & \text{худший случай}
		\end{cases},
	\end{equation}
	где $f_1$ --- трудоемкость блока, который вычисляется при выполнении условия, а $f_2$ --- трудоемкость блока, который вычисляется при невыполнении условия;
	\item трудоемкость цикла рассчитывается по формуле \eqref{eq:for},
	\begin{equation}
		\label{eq:for}
		\begin{gathered}
			f_{for} = f_{\text{инициализация}} + f_{\text{сравнения}} + M_{\text{итераций}} \cdot (f_{\text{тело}} +\\
			+ f_{\text{инкремент}} + f_{\text{сравнения}});
		\end{gathered}
	\end{equation}
	\item вызов подпрограмм и передача параметров принимается за $0$.
\end{enumerate}

\subsection{Трудоемкость алгоритма полного перебора}

Добавление в конец списка считается операцией стоимостью в $1$. Тогда трудоемкость алгоритма составления массива перестановок из $n$ элементов оценивается по формуле \eqref{eq:perms}.
\begin{equation}
	\label{eq:perms}
	f_{perm} = 9 \cdot n! + 3
\end{equation}

Трудоемкость расчета пути длины $n$ считается как $6n - 4$. Тогда трудоемкость алгоритма полного перебора в худшем случае считается по формуле \eqref{eq:brut}.
\begin{equation}
	\label{eq:brut}
	f_{brut} = (6n + 11)(n-1)! + 6n + 1 = O(n!)
\end{equation}


\subsection{Трудоемкость муравьиного алгоритма}

Трудоемкость рассчитывается для $n$ городов и $T$ дней. Трудоемкость алгоритма вычисления массива вероятностей переходов в города, рассчитывается по формуле \eqref{eq:pk}.
\begin{equation}
	\label{eq:pk}
	f_{pk} = 13n + 4
\end{equation}

Трудоемкость алгоритма обновления матрицы феромона, рассчитывается по формуле \eqref{eq:pher_update}.
\begin{equation}
	\label{eq:pher_update}
	f_{update} = 6n^4 + n^3 + 19n^2 + 4n + 3
\end{equation}

Тогда трудоемкость муравьиного алгоритма рассчитывается по формуле~\eqref{eq:ant}.
\begin{equation}
	\label{eq:ant}
	f_{ant} = T(6n^4 + 15n^3 + 25n^2 + 15n + 7) + 11n^2 + 14n + 7 = O(Tn^4)
\end{equation}

\section*{Вывод}
На основе теоретических данных, полученных из аналитического раздела были построены схемы требуемых алгоритмов. 
Была введена модель оценки трудоемкости алгоритма, были рассчитаны трудоемкости алгоритмов в соответствии с этой моделью.

В результате теоретической оценки трудоемкостей алгоритмов выяснилось, что лучшей асимптотической оценкой обладает муравьиный алгоритм $O(Tn^4)$, где $T$ --- количество дней жизни колонии, а $n$ --- количество городов. 
Алгоритм полного перебора обладает асимптотической оценкой $O(n!)$.