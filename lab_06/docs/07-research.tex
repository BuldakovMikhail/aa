\chapter{Исследовательский раздел}

В данном разделе будут приведены: пример работы программы, постановка эксперимента и сравнительный анализ алгоритмов на основе полученных данных.

\section{Демонстрация работы программы}


На рисунке \ref{img:program} представлена демонстрация работы разработанного программного обеспечения, а именно показаны результаты решения задачи коммивояжера для графа, заданного матрицей смежности $\begin{pmatrix}
	0 & 6 & 1 & 9 & 9 \\
	1 & 0 & 6 & 2 & 7 \\
	1 & 4 & 0 & 6 & 5 \\
	8 & 5 & 5 & 0 & 8 \\
	5 & 1 & 4 & 8 & 0 \\
\end{pmatrix}$ .  
\clearpage

\includeimage
{program} % Имя файла без расширения (файл должен быть расположен в директории inc/img/)
{f} % Обтекание (без обтекания)
{h} % Положение рисунка (см. figure из пакета float)
{1\textwidth} % Ширина рисунка
{Демонстрация работы программы при решении задачи коммивояжера} % Подпись рисунка

\clearpage


\section{Технические характеристики}

Технические характеристики устройства, на котором выполнялись замеры по времени, следующие:
\begin{itemize}
	\item процессор: AMD Ryzen 5 4600H 3 ГГц \cite{amd};
	\item оперативная память: 16 ГБайт;
	\item операционная система: Windows 10 Pro 64-разрядная система версии 22H2 \cite{windows}.
\end{itemize}

При замерах времени ноутбук был включен в сеть электропитания и был нагружен только системными приложениями.

\section{Время выполнения реализаций алгоритмов}

Результаты замеров времени выполнения реализаций алгоритмов решения задачи коммивояжера приведены в таблице \ref{tbl:time_measurements}.
Замеры времени проводились на полносвязных графах одного размера и усреднялись для каждого набора одинаковых экспериментов. Для реализации муравьиного алгоритма количество дней равно 10.

\begin{table}[h]
	\begin{center}
		\begin{threeparttable}
			\captionsetup{justification=raggedright,singlelinecheck=off}
			\caption{Время работы реализации алгоритмов решения задачи коммивояжера (в с)}
			\label{tbl:time_measurements}
			\begin{tabular}{|c|c|c|}
				\hline
				Количество городов &  Полный перебор  & Муравьиный \\
				\hline
				4 &$ 1.563\cdot10^{-5} $&$ 1.422\cdot10^{-3}$\\
				\hline
				5 &$ 4.688\cdot10^{-5} $&$ 3.125\cdot10^{-3}$\\
				\hline
				6 &$ 2.344\cdot10^{-4} $&$ 5.750\cdot10^{-3}$\\
				\hline
				7 &$ 1.594\cdot10^{-3} $&$ 1.005\cdot10^{-2}$\\
				\hline
				8 &$ 1.231\cdot10^{-2} $&$ 1.664\cdot10^{-2}$\\
				\hline
				9 &$ 1.105\cdot10^{-1} $&$ 2.506\cdot10^{-2}$\\
				\hline
			\end{tabular}
		\end{threeparttable}
	\end{center}
\end{table}

\section{Параметризация муравьиного алгоритма}

Автоматическая параметризация была проведена на двух классах данных --- \ref{par:class1} и \ref{par:class2}. Алгоритм будет запущен для набора значений $\alpha, \rho \in (0, 1)$ и $t_{max} \in \{5, 25, 50, 100\}$.

Итоговая таблица значений параметризации будет состоять из следующих колонок:
\begin{itemize}[label=---]
	\item $\alpha$ --- коэффициент жадности;
	\item $\rho$ --- коэффициент испарения;
	\item $t_{max}$ --- количество дней жизни колонии муравьев;
	\item \textit{Результат} --- максимальная длина пути, полученная муравьиным алгоритмом за 10 запусков;
	\item \textit{Ошибка} --- разность между эталонным значением и результатом.
\end{itemize}

\subsection{Класс данных 1}\label{par:class1}

Класс данных 1 представляет собой матрицу \eqref{eq:mt1} смежности размером 10 элементов со значениями от 1 до 100.

\begin{equation}
	\label{eq:mt1}
	M_1 = 
	\begin{pmatrix}
		0 & 34 & 84 & 64 & 37 & 51 & 7 & 55 & 28 & 10 \\
		34 & 0 & 89 & 61 & 26 & 64 & 31 & 82 & 19 & 48 \\
		11 & 16 & 0 & 46 & 2 & 75 & 48 & 65 & 45 & 62 \\
		12 & 30 & 46 & 0 & 71 & 37 & 27 & 70 & 44 & 25 \\
		68 & 20 & 31 & 36 & 0 & 47 & 44 & 72 & 29 & 82 \\
		90 & 78 & 11 & 44 & 91 & 0 & 62 & 43 & 73 & 77 \\
		90 & 33 & 80 & 8 & 98 & 48 & 0 & 99 & 36 & 71 \\
		18 & 16 & 28 & 22 & 99 & 62 & 80 & 0 & 31 & 63 \\
		51 & 77 & 45 & 91 & 45 & 41 & 77 & 40 & 0 & 26 \\
		55 & 67 & 24 & 8 & 57 & 29 & 82 & 50 & 78 & 0 \\
	\end{pmatrix}
\end{equation}

Результаты параметризации для первого класса данных содержатся в приложении \ref{tabel_mt1}.

\subsection{Класс данных 2}\label{par:class2}

Класс данных 1 представляет собой матрицу \eqref{eq:mt2} смежности размером 10 элементов со значениями от 100 до 10000.

\begin{equation}
	\label{eq:mt2}
	M_2 = 
	\begin{pmatrix}
		0 & 2807 & 2494 & 4820 & 8257 & 8688 & 2784 & 7073 & 5246 & 5816 \\
		6688 & 0 & 3579 & 5313 & 8630 & 1530 & 6084 & 6745 & 7040 & 9483 \\
		3070 & 1462 & 0 & 5040 & 8379 & 1145 & 9036 & 4213 & 8686 & 9060 \\
		7869 & 6450 & 5517 & 0 & 2936 & 3890 & 8459 & 1535 & 264 & 6446 \\
		7588 & 1538 & 4715 & 957 & 0 & 3324 & 8420 & 705 & 780 & 2114 \\
		5313 & 7012 & 6748 & 3221 & 8283 & 0 & 738 & 2073 & 1966 & 9134 \\
		3297 & 2578 & 8005 & 2220 & 4799 & 8012 & 0 & 5366 & 6696 & 1371 \\
		2885 & 6139 & 3617 & 5044 & 4018 & 1991 & 5720 & 0 & 3252 & 3749 \\
		6359 & 5162 & 2316 & 2605 & 3085 & 8213 & 8597 & 8510 & 0 & 9288 \\
		4739 & 7850 & 3077 & 222 & 3818 & 8252 & 8204 & 8576 & 4921 & 0 \\
	\end{pmatrix}
\end{equation}

Результаты параметризации для второго класса данных содержатся в приложении \ref{tabel_mt2}.