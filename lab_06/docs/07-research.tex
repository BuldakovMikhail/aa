\chapter{Исследовательский раздел}

В данном разделе будут приведены: пример работы программы, постановка эксперимента и сравнительный анализ алгоритмов на основе полученных данных.

\section{Демонстрация работы программы}

На рисунке \ref{img:program} представлена демонстрация работы разработанного программного обеспечения, а именно показаны результаты решения задачи коммивояжера для графа, заданного матрицей смежности $\begin{pmatrix}
	0 & 6 & 1 & 9 & 9 \\
	1 & 0 & 6 & 2 & 7 \\
	1 & 4 & 0 & 6 & 5 \\
	8 & 5 & 5 & 0 & 8 \\
	5 & 1 & 4 & 8 & 0 \\
\end{pmatrix}$.  
\clearpage

\includeimage
{program} % Имя файла без расширения (файл должен быть расположен в директории inc/img/)
{f} % Обтекание (без обтекания)
{h} % Положение рисунка (см. figure из пакета float)
{1\textwidth} % Ширина рисунка
{Демонстрация работы программы при решении задачи коммивояжера} % Подпись рисунка

\clearpage


\section{Технические характеристики}

Технические характеристики устройства, на котором выполнялись замеры по времени, следующие:
\begin{itemize}
	\item процессор: AMD Ryzen 5 4600H 3 ГГц~\cite{amd};
	\item оперативная память: 16 ГБайт;
	\item операционная система: Windows 10 Pro 64-разрядная система версии 22H2~\cite{windows}.
\end{itemize}

При замерах времени ноутбук был включен в сеть электропитания и был нагружен только системными приложениями.

\section{Время выполнения реализаций алгоритмов}

Результаты замеров времени выполнения реализаций алгоритмов решения задачи коммивояжера приведены в таблице \ref{tbl:time_measurements}.
Замеры времени проводились на полносвязных графах одного размера и усреднялись для каждого набора одинаковых экспериментов. Для реализации муравьиного алгоритма количество дней равно 10.

\begin{table}[h]
	\begin{center}
		\begin{threeparttable}
			\captionsetup{justification=raggedright,singlelinecheck=off}
			\caption{Время работы реализации алгоритмов решения задачи коммивояжера (в с)}
			\label{tbl:time_measurements}
			\begin{tabular}{|c|c|c|}
				\hline
				Количество городов &  Полный перебор  & Муравьиный \\
				\hline
				4 &$ 1.563\cdot10^{-5} $&$ 1.422\cdot10^{-3}$\\
				\hline
				5 &$ 4.688\cdot10^{-5} $&$ 3.125\cdot10^{-3}$\\
				\hline
				6 &$ 2.344\cdot10^{-4} $&$ 5.750\cdot10^{-3}$\\
				\hline
				7 &$ 1.594\cdot10^{-3} $&$ 1.005\cdot10^{-2}$\\
				\hline
				8 &$ 1.231\cdot10^{-2} $&$ 1.664\cdot10^{-2}$\\
				\hline
				9 &$ 1.105\cdot10^{-1} $&$ 2.506\cdot10^{-2}$\\
				\hline
			\end{tabular}
		\end{threeparttable}
	\end{center}
\end{table}

\clearpage
На рисунках \ref{img:compare} и \ref{img:compare1} изображены графики зависимостей времени выполнения реализаций алгоритмов решения задачи коммивояжера от количества городов.

\includeimage
{compare} % Имя файла без расширения (файл должен быть расположен в директории inc/img/)
{f} % Обтекание (без обтекания)
{h} % Положение рисунка (см. figure из пакета float)
{1\textwidth} % Ширина рисунка
{Сравнение реализаций алгоритмов по времени выполнения} % Подпись рисунка

\includeimage
{compare1} % Имя файла без расширения (файл должен быть расположен в директории inc/img/)
{f} % Обтекание (без обтекания)
{h} % Положение рисунка (см. figure из пакета float)
{1\textwidth} % Ширина рисунка
{Сравнение реализаций алгоритмов по времени выполнения} % Подпись рисунка

\clearpage

\section{Параметризация муравьиного алгоритма}

Автоматическая параметризация была проведена для одного класса данных \ref{par:class1}, состоящего из 3 графов. Алгоритм запускался для набора значений $\alpha, \rho \in (0, 1)$ и $t_{max} \in \{5, 25, 50, 100\}$.

Итоговая таблица значений параметризации состоит из следующих колонок:
\begin{itemize}
	\item $\alpha$ --- коэффициент жадности;
	\item $\rho$ --- коэффициент испарения феромона;
	\item $t_{max}$ --- количество дней жизни колонии муравьев;
	\item \textit{Результат} --- максимальная длина пути, полученная муравьиным алгоритмом за 10 запусков;
	\item \textit{Ошибка} --- разность между результатом и эталонным значением.
\end{itemize}

\subsection{Класс данных}\label{par:class1}

Класс данных представляет собой набор из 3 орграфов, заданных с помощью матриц смежности \eqref{eq:mt1}, \eqref{eq:mt2}, \eqref{eq:mt3} размером 10 элементов с равномерным распределением значений весов от 1 до 100.

\begin{equation}
	\label{eq:mt1}
	M_1 = 
	\begin{pmatrix}
		0 & 34 & 84 & 64 & 37 & 51 & 7 & 55 & 28 & 10 \\
		34 & 0 & 89 & 61 & 26 & 64 & 31 & 82 & 19 & 48 \\
		11 & 16 & 0 & 46 & 2 & 75 & 48 & 65 & 45 & 62 \\
		12 & 30 & 46 & 0 & 71 & 37 & 27 & 70 & 44 & 25 \\
		68 & 20 & 31 & 36 & 0 & 47 & 44 & 72 & 29 & 82 \\
		90 & 78 & 11 & 44 & 91 & 0 & 62 & 43 & 73 & 77 \\
		90 & 33 & 80 & 8 & 98 & 48 & 0 & 99 & 36 & 71 \\
		18 & 16 & 28 & 22 & 99 & 62 & 80 & 0 & 31 & 63 \\
		51 & 77 & 45 & 91 & 45 & 41 & 77 & 40 & 0 & 26 \\
		55 & 67 & 24 & 8 & 57 & 29 & 82 & 50 & 78 & 0 \\
	\end{pmatrix}
\end{equation}

\begin{equation}
	\label{eq:mt2}
	M_2 = 
	\begin{pmatrix}
		0 & 73 & 27 & 98 & 40 & 71 & 11 & 31 & 17 & 14 \\
		5 & 0 & 19 & 99 & 74 & 29 & 43 & 54 & 94 & 58 \\
		78 & 66 & 0 & 17 & 79 & 98 & 74 & 64 & 39 & 45 \\
		4 & 83 & 56 & 0 & 71 & 28 & 32 & 50 & 96 & 11 \\
		36 & 62 & 51 & 35 & 0 & 51 & 62 & 43 & 39 & 94 \\
		6 & 3 & 53 & 23 & 42 & 0 & 10 & 9 & 46 & 10 \\
		25 & 82 & 23 & 50 & 32 & 65 & 0 & 72 & 69 & 65 \\
		53 & 93 & 55 & 62 & 71 & 42 & 79 & 0 & 79 & 78 \\
		65 & 55 & 45 & 17 & 44 & 82 & 68 & 32 & 0 & 48 \\
		93 & 70 & 2 & 53 & 62 & 27 & 76 & 25 & 81 & 0 \\
	\end{pmatrix}
\end{equation}

\begin{equation}
	\label{eq:mt3}
	M_3 = 
	\begin{pmatrix}
		0 & 22 & 97 & 48 & 76 & 64 & 32 & 66 & 63 & 50 \\
		34 & 0 & 69 & 46 & 13 & 63 & 71 & 99 & 48 & 83 \\
		74 & 21 & 0 & 3 & 64 & 49 & 78 & 38 & 61 & 42 \\
		86 & 4 & 40 & 0 & 57 & 16 & 74 & 36 & 99 & 98 \\
		92 & 51 & 98 & 42 & 0 & 38 & 19 & 28 & 19 & 18 \\
		29 & 92 & 47 & 30 & 99 & 0 & 33 & 86 & 51 & 4 \\
		11 & 9 & 94 & 46 & 31 & 5 & 0 & 46 & 55 & 45 \\
		52 & 95 & 13 & 3 & 19 & 25 & 77 & 0 & 75 & 14 \\
		64 & 49 & 25 & 36 & 4 & 96 & 46 & 50 & 0 & 61 \\
		17 & 54 & 68 & 80 & 81 & 84 & 93 & 35 & 94 & 0 \\
	\end{pmatrix}
\end{equation}


Результаты параметризации для  класса данных содержатся в приложении \ref{tabel_mt1}.


\section*{Вывод}

В результате замеров времени выполнения реализаций алгоритмов было выявлено, что реализация алгоритма полного перебора оказалась быстрее реализации муравьиного алгоритма при количестве городов меньше 8, например, при количестве городов 4, реализация алгоритма полного перебора выигрывает реализацию муравьиного алгоритма в 91 раз. 
Но при количестве городов 9, реализация алгоритма полного перебора оказалась хуже реализации муравьиного алгоритма в 4 раза по времени выполнения.
Что соответствует асимптотическим оценкам трудоемкости алгоритмов, а именно алгоритм полного перебора обладает большей асимптотической оценкой $O(n!)$, чем муравьиный алгоритм $O(Tn^4)$.
Т.~о. алгоритм полного перебора выгоднее применять при малом количестве городов.

В результате параметризации были подобраны параметры, при которых реализация муравьиного алгоритма показывает наилучшие результаты. Для первого графа такими параметрами являются: $\alpha = 0.1, \rho = \{0.1, 0.3\}, t_{max}=100$. Для второго: $\alpha = 0.1, \rho = 0.9, t_{max}=100$. Для третьего:
\begin{itemize}
	\item $\alpha = 0.1, \rho = 0.1, t_{max}=\{50, 100\}$;
	\item $\alpha = 0.1, \rho = 0.3, t_{max}=100$;
	\item $\alpha = 0.1, \rho = 0.5, t_{max}=\{50, 100\}$;
	\item $\alpha = 0.1, \rho = 0.7, t_{max}=100$;
	\item $\alpha = 0.1, \rho = 0.9, t_{max}=100$.
\end{itemize}

Т.~о. для реализации муравьиного алгоритма, применяемого к рассмотренному классу данных, рекомендуется использовать параметры: $\alpha = 0.1$, $\rho = 0.1$, $t_{max} = 100$.