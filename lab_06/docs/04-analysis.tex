\chapter{Аналитический раздел}

В данном разделе будут описаны алгоритмы решения задачи коммивояжера.

\section{Описание задачи}

Задача коммивояжера --- одна из самых известных задач комбинаторной оптимизации, заключающаяся в поиске наиболее выгодного маршрута, проходящего через указанные города и возвращающегося в начальный пункт.

В общем случае задача формулируется следующим образом: имеется $N$ городов, для каждой пары которых известно расстояние между ними. Требуется найти такой маршрут, проходящий через каждый город по одному разу (и возвращающийся в исходный город), при этом сумма всех расстояний на этом маршруте должна быть наименьшей из возможных \cite{intro}.

В данной работе будет рассматриваться вариант задачи с незамкнутым маршрутом, т.~е. без одного последнего перехода.
Стоимость пути между городами будет отличаться в различных направлениях, поэтому граф будет ориентированным.

\section{Алгоритмы решения задачи}

\textbf{Алгоритм полного перебора}

Задача может быть решена перебором всех вариантов объезда и выбором оптимального.
Очевидно, что при полном переборе будет найден самый кратчайший маршрут, но при этом для перебора необходимо будет выполнить порядка $O(N!)$ операций, где $N$ --- количество городов, что является тяжелой задачей даже для современных ЭВМ при $N$ порядка сотни.

\textbf{Муравьиный алгоритм (без элитных муравьев)}

Муравьиный алгоритм --- алгоритм решения задачи коммивояжера, основанный на принципе поведения колонии
муравьев \cite{ants}. 

Муравьи действуют, руководствуясь органами чувств. Каждый муравей оставляет на своем пути феромоны, чтобы другие могли ориентироваться. При большом количестве муравьев наибольшее количество феромона остается на наиболее посещаемом пути, посещаемость же может быть связана с длинами ребер. Муравьи используют непрямой обмен информацией через окружающую среду посредством феромона.

Основная идея заключается в том, что выделяются две фазы: день и ночь. В фазу дня каждый муравей $k$ строит один маршрут, вечером обновляется лучшая траектория. В фазу ночи обновляется матрица феромона. 

Каждый муравей имеет 3 способности:
\begin{enumerate}
	\item Зрение --- муравей $k$, стоя в городе $i$, может оценить привлекательность ребра $i$ --- $j$;
	\item Обоняние --- муравей чует концентрацию феромона $\tau_{ij}(t)$ на ребре $i$---$j$ в текущий день $t$;
	\item Память --- муравей запоминает список, посещенных за текущий день $t$ городов --- $J_k(t)$.
\end{enumerate}

Привлекательность ребра $i$---$j$ оценивается по формуле \eqref{eq:likely}.
\begin{equation}
	\label{eq:likely}
	\eta_{ij} = \frac{1}{D_{ij}},
\end{equation}
где $D_{ij}$ --- метка ребра, $D$ --- матрица смежности.

Стоя в городе $i$, муравей $k$ выбирает следующий город на основе вероятностного правила \eqref{eq:prob_rule}.
\begin{equation}
	\label{eq:prob_rule}
	p_{ij, k} = \begin{cases}
		0, j \in J_k, \\
		\frac{\eta_{ij}^{\alpha}\cdot\tau_{ij}^{\beta}(t)}{\sum_{q\in J_k} \eta^\alpha_{iq}\cdot\tau^\beta_{iq}(t)}, \text{иначе},
	\end{cases}
\end{equation}
где $\alpha$ --- коэффициент жадности, $\beta$ --- коэффициент стадности, причем $\alpha + \beta = 1$.

После завершения движения всех муравьев (ночью, перед наступлением следующего дня), феромон обновляется по формуле \eqref{update_phero_1}.
\begin{equation}
	\label{update_phero_1}
	\tau_{ij}(t+1) = \tau_{ij}(t)\cdot(1-\rho) + \Delta \tau_{ij}(t),
\end{equation}
где $\rho \in [0, 1]$ --- коэффициент испарения феромона. 

\begin{equation}
	\label{update_phero_2}
	\Delta \tau_{ij}(t) = \sum_{k=1}^N \Delta \tau_{ij, k}(t).
\end{equation}

\begin{equation}
	\label{update_phero_3}
	\Delta\tau^k_{ij}(t) = \begin{cases}
		0, \textrm{муравей k в день t не ходил по ребру i---j,} \\
		Q/L_{k}, \textrm{иначе},
	\end{cases}
\end{equation}
где $Q$ --- квота феромона одного муравья на день, $Q$ выбирается соразмерной длине лучшего маршрута в графе.

Чтобы значение феромона не обнулилось и не повлекло обнуление вероятности перехода по ребру, после расчета значения $\tau_{ij}(t+1)$ необходимо выполнять проверку матрицы феромона и все значения, которые меньше заданного порога, заменить на порог.

В данном алгоритме отсутствует полный перебор, что означает меньшую трудоемкость, чем для алгоритма полного перебора, но при этом лучшее решение не гарантируется.
 

\section*{Вывод}

В данном разделе были описаны алгоритм полного перебора и муравьиный алгоритм для решения задачи коммивояжера.




