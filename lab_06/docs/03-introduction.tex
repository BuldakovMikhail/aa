\chapter*{ВВЕДЕНИЕ}
\addcontentsline{toc}{chapter}{ВВЕДЕНИЕ}

Задача коммивояжера является одной из классических задач комбинаторной оптимизации, привлекающей внимание исследователей и практиков в области логистики, транспорта и информационных технологий.
В контексте логистики, эта задача применяется для оптимизации маршрутов доставки товаров и грузов, что позволяет сократить затраты на перевозку, уменьшить время доставки и улучшить эффективность работы логистических компаний~\cite{intro}.

Цель данной лабораторной работы --- рассмотреть алгоритмы решения задачи коммивояжера в случае построения карты перемещений для воздухоплавателей.

Для достижения поставленной цели необходимо выполнить следующие задачи:
\begin{itemize}
	\item описать алгоритмы решения задачи коммивояжера;
	\item спроектировать программное обеспечение, реализующее алгоритмы решения задачи коммивояжера;
	\item выбрать инструменты для реализации и замера процессорного времени
	выполнения реализаций решения задачи;
	\item проанализировать затраты реализаций алгоритмов по времени.
\end{itemize}

