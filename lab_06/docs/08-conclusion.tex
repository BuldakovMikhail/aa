\chapter*{ЗАКЛЮЧЕНИЕ}
\addcontentsline{toc}{chapter}{ЗАКЛЮЧЕНИЕ}

Цель данной лабораторной работы была достигнута, а именно были рассмотрены алгоритмы решения задачи коммивояжера.

Для достижения поставленной цели были выполнены следующие задачи.
\begin{itemize}
	\item описаны алгоритмы решения задачи коммивояжера;
	\item спроектировано программное обеспечение, реализующее алгоритмы решения задачи коммивояжера;
	\item выбраны инструменты для реализации и замера процессорного времени
	выполнения реализаций решения задачи;
	\item проанализированы затраты реализаций алгоритмов по времени.
\end{itemize}

В результате исследования времени выполнения реализаций алгоритмов было выявлено, что реализация алгоритма полного перебора оказалась быстрее реализации муравьиного алгоритма при количестве городов меньше 8, например, при количестве городов 4, реализация алгоритма полного перебора выигрывает реализацию муравьиного алгоритма в 91 раз по времени выполнения. 
Но при количестве городов 9, реализация алгоритма полного перебора оказалась хуже реализации муравьиного алгоритма в 4 раза по времени выполнения.
Что соответствует асимптотическим оценкам трудоемкости алгоритмов, а именно алгоритм полного перебора обладает большей асимптотической оценкой $O(n!)$, чем муравьиный алгоритм $O(Tn^4)$.

В ходе параметризации было выявлено, что лучший результат, реализация муравьиного алгоритма, достигает при значениях параметров: $\alpha = 0.1$, $\rho = 0.1$, $t_{max} = 100$.